%\DocumentMetadata{testphase=phase-III}
\documentclass[12pt,a4paper]{article}
\usepackage{amsmath,amssymb,amsthm,booktabs,cancel,color,enumitem,graphicx,listings,mathtools,metalogo,%tikz,mathrsfs,
multicol,pdfpages,pgfplots,rotating,subfig,widebar,xparse,xspace}

\pgfplotsset{compat=1.15}
\usetikzlibrary{intersections}

%\usepackage{eucal}
\usepackage[scr]{rsfso}
\usepackage[margin=1in]{geometry}
\usepackage[comma,longnamesfirst,round]{natbib}
%
\usepackage[colorlinks=true,% If false then links black, but set in ugly red boxes
                            % If true then the following colour scheme is used.
            citecolor=blue,% The named colours include: black, white, red, yellow
            linkcolor=red,%                             blue, green, cyan, magenta
            anchorcolor=yellow,%
            filecolor=cyan,% Can set all all colors to a single color using the following command
            menucolor=red,% allcolors=<colour>.  For example, allcolors=black would be idea for
            runcolor=filecolor,% printing hard copies.  Simply uncomment the command on the following
            urlcolor=magenta,% line if you want this.
            %allcolors=black,
            pagebackref=true% In the bibliography gives citation page references. To turn on set
                            % to true
            ]{hyperref}

%% 
 % \usepackage[tagged, highstructure]{accessibility}
 % \usepackage{axessibility}
 %%
            
%
% Theorems and Remarks
%
\theoremstyle{definition}
\newtheorem{ass}{Assumption}

\theoremstyle{plain}
\newtheorem*{Thm}{Theorem}
\newtheorem{thm}{Theorem}
\theoremstyle{remark}
\newtheorem*{remark}{Remark}
\newtheorem{rem}{Remark}
\newtheorem{corr}{Corollary}
\theoremstyle{definition}
%\newtheorem{Def}{Definition}
\newtheorem*{deff}{Definition}

\newtheoremstyle{example}% name
  {9pt}%      Space above, empty = `usual value'
  {9pt}%      Space below
  {\normalfont} %\renewcommand{\myfiletype}{loe}}% Body font
  {}%         Indent amount (empty = no indent, \parindent = para indent)
  {\bfseries}% Thm head font
  {.}%        Punctuation after thm head
  {\newline}% Space after thm head: \newline = linebreak
  {\thmname{#1}\thmnumber{ #2.}\thmnote{ #3}}%Thm head spec
  
\theoremstyle{example}
\newtheorem{example}{Example}

\newtheoremstyle{Def}% name
  {9pt}%      Space above, empty = `usual value'
  {9pt}%      Space below
  {\normalfont}%\addcontentsline{lod}{Def}{\protect\numberline{\theDef}y}} %\renewcommand{\myfiletype}{lod}}% Body font
  {}%         Indent amount (empty = no indent, \parindent = para indent)
  {\bfseries}% Thm head font
  {.}%        Punctuation after thm head
  {\newline}% Space after thm head: \newline = linebreak
  {\thmname{#1}\thmnumber{ #2.}\thmnote{ #3}}%         Thm head spec
  
\theoremstyle{Def}
\newtheorem{Def}{Definition}

\newcommand{\abs}[1]{\ensuremath{\operatorname{abs}\left(#1\right)}}
\newcommand{\ans}{\newline\textit{\textcolor{red}{Answer:}}\newline}
\newcommand{\ant}{\textit{\textcolor{red}{Answer:}}\newline}
\makeatletter
   \def\argmax{\mathop{\operator@font argmax}}
   \def\argmin{\mathop{\operator@font argmin}}
\makeatother
%\newcommand{\argmax}{\arg\max}
%\newcommand{\argmin}{\arg\min}
\newcommand{\Beta}[2]{\ensuremath{\operatorname{B}\left(#1,#2\right)}}
\def\BibTeX{{\rm B\kern-.05em{\sc i\kern-.025em b}\kern-.08em
    T\kern-.1667em\lower.7ex\hbox{E}\kern-.125emX}}
\newcommand{\cmd}[1]{\mbox{\texttt{\symbol{'134}#1}}}
\newcommand{\Co}[1]{\ensuremath \mathrm{Cov}_{0}\left[#1\right]}
\newcommand{\cor}[2]{\ensuremath{\operatorname{cor}\left(#1,#2\right)}}
\newcommand{\cov}[2]{\ensuremath{\operatorname{Cov}\left[#1,#2\right]}}
\DeclareMathOperator{\diag}{diag}
\newcommand{\dr}{\ensuremath{\mrmd r}}
\newcommand{\ds}{\ensuremath{\mrmd s}}
\newcommand{\dt}{\ensuremath{\mrmd t}}
\newcommand{\dw}{\ensuremath{\mrmd w}}
\newcommand{\dx}{\ensuremath{\mrmd x}}
\newcommand{\dy}{\ensuremath{\mrmd y}}
\newcommand{\dz}{\ensuremath{\mrmd z}}
\DeclareMathOperator{\etr}{etr}
\newcommand{\E}[1]{\ensuremath{\operatorname{E}\left[#1\right]}}
\newcommand{\Eo}[1]{\ensuremath \mathrm{E}_{0}\left[#1\right]}
\newcommand{\Eq}[1]{\begin{align*}#1\end{align*}}
\newcommand{\Ex}[2]{\ensuremath{\operatorname{E}_{#1}\left[#2\right]}}
\newcommand{\Gammafrac}[2][2]{\ensuremath{\Gamma\left(\tfrac{#2}{#1}\right)}}%\Gamma[x]{y} yields Gamma(y/x)
\newcommand{\gfn}[1]{\ensuremath{\Gamma\left(#1\right)}}
\newcommand{\half}[1]{\ensuremath{\frac{#1}{2}}}
\newcommand{\hyperg}[2]{\ensuremath{\prescript{}{#1}{F}^{}_{#2}}}
\newcommand{\im}[1]{\ensuremath{\operatorname{Im}\left(#1\right)}}
\newcommand{\intinfty}{\int^{\infty}_{0}}
\newcommand{\matlab}{\textsc{Matlab}\xspace}
\newcommand{\mg}{\textit{\textcolor{red}{Marking Guide}}:\xspace\newline}
\newcommand{\mgt}{\newline\textit{\textcolor{blue}{Marking Guide:}}\newline}
\newcommand{\Mo}[1]{\ensuremath \mathrm{MSE}_{0}\left[#1\right]}
\newcommand{\mrmd}{\ensuremath{\,\mathrm{d}}}
\newcommand{\myrego}{\,\raisebox{1.5pt}{\scalebox{0.81}{\textregistered}}\xspace}
\DeclareMathOperator{\N}{N}
\newcommand{\ninf}[1][n]{#1\to\infty}
\newcommand{\param}[2][n]{\ensuremath{#2_{1} #2_{2}\ldots #2_{#1}}}
\newcommand{\params}[2][n]{\ensuremath{#2_{1},#2_{2},\ldots,#2_{#1}}}
\newcommand{\paramsj}[3][j]{\ensuremath{\left(#2_{1}\right)_{#1}\left(#2_{2}\right)_{#1}\ldots\left(#2_{#3}\right)_{#1}}}
\newcommand{\pdflatex}{pdf\LaTeX\xspace}
\makeatletter
   \def\plim{\mathop{\operator@font plim}}
\makeatother
\newcommand{\Poch}[2]{\ensuremath{\left(#1\right)_{#2}}}
\newcommand{\Prob}[1]{\ensuremath{\operatorname{Pr}\left(#1\right)}}
\newcommand{\re}[1]{\ensuremath{\operatorname{Re}\left(#1\right)}}
\newcommand{\sgn}[1]{\ensuremath{\operatorname{sgn}\left(#1\right)}}
%Duplicated commands:  Settle on a choice.  No. They do different things!
\newcommand{\solution}{\textit{\textcolor{red}{Solution}}:\xspace\newline}
\newcommand{\sol}{\textcolor{red}{Solution:}\xspace}
%
\DeclareDocumentCommand{\sumj}{ O{j} O{0} }{\ensuremath{\sum_{#1=#2}^{\infty}\,}}
\newcommand{\tod}{\ensuremath{\overset{d}{\to}}}
\DeclareMathOperator{\tr}{tr}
\DeclareMathOperator{\myvec}{vec}
\DeclareMathOperator{\vech}{vech}
% More duplicates: Choice as been made?
\newcommand{\V}[1]{\ensuremath{\operatorname{Var}\left[#1\right]}}
\newcommand{\Vhat}[1]{\ensuremath{\operatorname{\protect\widehat{V}}\left[#1\right]}}
\newcommand{\Vo}[1]{\ensuremath \mathrm{V}_{0}\left[#1\right]}
%\newcommand{\var}[1]{\ensuremath{\operatorname{Var}\left(#1\right)}}
\newcommand{\Vx}[2]{\ensuremath{\operatorname{Var}_{#1}\left[#2\right]}}

%Boxes
\newsavebox{\mystrutbox}
\savebox{\mystrutbox}{\rule{0mm}{0.6cm}}
\newcommand{\mystrut}{\usebox{\mystrutbox}}

\newlength{\twosuper}
\settowidth{\twosuper}{\textsuperscript{2}}

%
% List Management
%
\makeatletter
   \renewcommand{\p@enumii}{\relax}
   \def\plim{\mathop{\operator@font plim}}
\makeatother
\renewcommand{\theenumii}{(\alph{enumii})}
\renewcommand{\labelenumii}{\theenumii}
\makeatletter
   \renewcommand{\p@enumii}{\theenumi}
   \renewcommand{\p@enumiii}{\theenumi\theenumii}
\makeatother

% 
% An extensions to amsmath matrix environments
% (adds the same column specification options that array has)
% originally from http://texblog.net/latex-archive/maths/amsmath-matrix/
% 
\makeatletter
\renewcommand*\env@matrix[1][*\c@MaxMatrixCols c]{%
  \hskip -\arraycolsep
  \let\@ifnextchar\new@ifnextchar
  \array{#1}}
\makeatother


\thispagestyle{empty}
\bibliographystyle{Skeels}

\title{ECOM40006/ECOM90013 Econometrics 3 \\
        Department of Economics \\
        University of Melbourne}
\date{Semester 1, \the\year}


\title{Week 2 Solutions}

\begin{document}

\maketitle

\section*{Computer Laboratory Session Activity}
R offers two commands \texttt{qqplot} for the construction of quantile-quantile plots:
\texttt{qqnorm} and \texttt{qqplot}.
Quantile-quantile plots, or Q-Q plots, as they are known, are a graphical device often used for
comparing the quantiles of an empirical distribution with those of a theoretical distribution.
Construct Q-Q plots comparing the following distributions against a t distribution with 5 degrees
of freedom: N(0,1), \( t_{20} \), \( t_{10} \), \( t_{5} \), \( t_{2} \), and \( t_{1} \).
Describe the patterns that you observe and the lessons that you can take from the exercise.\bigskip

\noindent\solution

\noindent First, a few of comments.
\begin{enumerate}
   \item I said in Lecture 1 that I would sometimes use the tutorials as an opportunity for you to
         teach yourself something useful that does not fit well into the lecture scenario.  Q-Q
         plots is one of those topics: really useful and a good opportunity to think about exactly
         what a distribution function is telling you, what quantiles are, and about probability in
         general.  My observation is that, as a group, you are not uniformly strong in your
         understanding of the probability that was taught in first year.  So this is a good chance
         to brush up on that.

   \item Daniel has provided a file, \texttt{Week2.R}, which covers the computing pretty well. Below
         I provide some discussion of what you might observe.  I need to issue one disclaimer.  I
         use Matlab much more than I use R and so the figures provided below were actually
         constructed in Matlab.  They were then converted into tikz, which is a graphical format
         that LaTeX can deal with, using the Matlab add-in matlab2tikz.

   \item Finally, I won't give you the \texttt{.tex} files used to produce all the Exercises and
         Solutions as a matter of course but, if you ever want to see how something is done then just
         ask.
\end{enumerate}

And now for the actual solution.  Q-Q plots plot the quantiles of one distribution against those of
some other distribution.  Usually one of the sets of quantiles are those from some theoretical
distribution, with the other set being from an empirical distribution.  This need not be the case,
as demonstrated by this exercise, where we are plotting one set of theoretical quantiles against
another.  The Q-Q plots originally requested can be found in Figures
\ref{fig:normqq}--\ref{fig:t1qq}.  As a bonus, Figures \ref{fig:chisq3qq} and \ref{fig:chisq30qq}
plot the theoretical quantiles of the distributions of centred chi-squared random variables against
those of the \( t_{5} \) distribution.  In particular, these figures compare the quantiles of the
distributions of random variables \( X \) against those of a \( t_{5} \) distribution where \(
X=\chi^{2}_{3}-3 \) and \( X=\chi^{2}_{30}-30 \), respectively.  These are presented because all the
other distributions considered are symmetric, whereas these distributions are skewed (to the right).
As a further exercise you should explore what happens if one of the distributions is skewed to the
left.  Figures \ref{fig:nct5qq} and \ref{fig:scaledt5qq} look at what happens when one plots the
quantiles of a distribution against those of a transformed version of the same underlying random
variable.  In particular, in Figure \ref{fig:nct5qq} we add a non-zero mean to the \( t_{5} \)
random variable and in Figure \ref{fig:scaledt5qq} we scale the variable by a constant.  As a final
note, the most common Q-Q plots have standard Normal distributions as the basis of comparison.  This
exercise illustrates the point that this need not be the case.
\begin{figure}[tbp]
   \centering
   % This file was created by matlab2tikz. 
%
%The latest updates can be retrieved from
%  http://www.mathworks.com/matlabcentral/fileexchange/22022-matlab2tikz-matlab2tikz
%where you can also make suggestions and rate matlab2tikz.
%
\definecolor{mycolor1}{rgb}{0.00000,0.44700,0.74100}%
%
\begin{tikzpicture}[scale=0.5]

\begin{axis}[%
width=6.028in,
height=4.754in,
at={(1.011in,0.642in)},
scale only axis,
xmin=-4,
xmax=4,
xlabel={\( t_{5} \) Quantiles},
ymin=-4,
ymax=4,
ylabel={Standard Normal Quantiles},
axis background/.style={fill=white},
axis x line*=bottom,
axis y line*=left,
legend style={at={(0.22,0.8)},anchor=south west,legend cell align=left,align=left,draw=white!15!black}
]
\addplot [color=red,solid]%dashdotted,forget plot
  table[row sep=crcr]{%
-3.36492999890722	-3.1249312410537\\
3.36492999890722	3.1249312410537\\
};
\addlegendentry{line passing through 1st and 3rd quartiles};

%% 
 % \addplot [color=red,solid,forget plot]
 %   table[row sep=crcr]{%
 % -0.717907367012227    -0.666703663995291\\
 % 0.717907367012227    0.666703663995291\\
 % };
 %%

\addplot [color=mycolor1,only marks,mark=+,mark options={solid,draw=blue}]%,forget plot
  table[row sep=crcr]{%
-3.36492999890722	-2.32634787404084\\
-2.75650852190947	-2.05374891063182\\
-2.42158470939533	-1.88079360815125\\
-2.19095825718118	-1.75068607125217\\
-2.01504837333302	-1.64485362695147\\
-1.87267775907536	-1.55477359459685\\
-1.75289015787549	-1.47579102817917\\
-1.64930035371341	-1.40507156030963\\
-1.55786934982539	-1.34075503369022\\
-1.47588404882448	-1.2815515655446\\
-1.40143609274796	-1.22652812003661\\
-1.33313328327202	-1.17498679206609\\
-1.26992940170539	-1.1263911290388\\
-1.21101864209398	-1.08031934081496\\
-1.15576734289429	-1.03643338949379\\
-1.10366827295606	-0.994457883209753\\
-1.05430909929714	-0.954165253146194\\
-1.00735007450321	-0.915365087842814\\
-0.962507893023184	-0.877896295051229\\
-0.919543780240827	-0.841621233572914\\
-0.878254550923454	-0.80642124701824\\
-0.838465792122455	-0.772193214188685\\
-0.800026592971978	-0.738846849185214\\
-0.762805418784888	-0.706302562840088\\
-0.726686843800421	-0.674489750196082\\
-0.691568936647646	-0.643345405392917\\
-0.657361147874823	-0.612812991016627\\
-0.623982587858245	-0.582841507271216\\
-0.591360611268586	-0.553384719555673\\
-0.559429644469353	-0.524400512708041\\
-0.528130207043046	-0.495850347347453\\
-0.497408089641652	-0.467698799114508\\
-0.467213658611261	-0.439913165673234\\
-0.437501264092247	-0.412463129441405\\
-0.408228733076414	-0.385320466407568\\
-0.379356932587402	-0.358458793251194\\
-0.35084939101406	-0.331853346436817\\
-0.322671967867415	-0.305480788099397\\
-0.294792563997342	-0.279319034447454\\
-0.267180865704145	-0.2533471031358\\
-0.239808117294835	-0.227544976641149\\
-0.212646917525774	-0.201893479141851\\
-0.185671036089451	-0.176374164780861\\
-0.158855246879451	-0.150969215496777\\
-0.132175175231687	-0.125661346855074\\
-0.105607156713064	-0.10043372051147\\
-0.0791281053272401	-0.0752698620998298\\
-0.0527153892438813	-0.0501535834647335\\
-0.0263467123422732	-0.0250689082587111\\
0	0\\
0.0263467123422732	0.0250689082587111\\
0.0527153892438814	0.0501535834647337\\
0.0791281053272402	0.0752698620998299\\
0.105607156713064	0.10043372051147\\
0.132175175231687	0.125661346855074\\
0.158855246879451	0.150969215496777\\
0.185671036089451	0.176374164780862\\
0.212646917525774	0.201893479141851\\
0.239808117294835	0.227544976641149\\
0.267180865704145	0.2533471031358\\
0.294792563997342	0.279319034447454\\
0.322671967867415	0.305480788099397\\
0.35084939101406	0.331853346436817\\
0.379356932587402	0.358458793251194\\
0.408228733076413	0.385320466407567\\
0.437501264092247	0.412463129441405\\
0.467213658611261	0.439913165673234\\
0.497408089641652	0.467698799114508\\
0.528130207043046	0.495850347347453\\
0.559429644469353	0.524400512708041\\
0.591360611268586	0.553384719555673\\
0.623982587858245	0.582841507271216\\
0.657361147874823	0.612812991016627\\
0.691568936647646	0.643345405392917\\
0.726686843800421	0.674489750196082\\
0.762805418784888	0.706302562840088\\
0.800026592971978	0.738846849185214\\
0.838465792122455	0.772193214188685\\
0.878254550923455	0.80642124701824\\
0.919543780240826	0.841621233572914\\
0.962507893023184	0.877896295051229\\
1.00735007450321	0.915365087842814\\
1.05430909929714	0.954165253146194\\
1.10366827295606	0.994457883209753\\
1.15576734289429	1.03643338949379\\
1.21101864209398	1.08031934081496\\
1.26992940170539	1.1263911290388\\
1.33313328327202	1.17498679206609\\
1.40143609274796	1.22652812003661\\
1.47588404882448	1.2815515655446\\
1.55786934982539	1.34075503369022\\
1.64930035371341	1.40507156030963\\
1.75289015787549	1.47579102817917\\
1.87267775907536	1.55477359459685\\
2.01504837333302	1.64485362695147\\
2.19095825718117	1.75068607125217\\
2.42158470939533	1.88079360815125\\
2.75650852190947	2.05374891063182\\
3.36492999890722	2.32634787404084\\
};
\addlegendentry{Q-Q Plot};

\end{axis}
\end{tikzpicture}%
   \caption{Q-Q Plot: \( \N(0,1) \) (or \( t_{\infty} \)) versus \( t_{5} \)}
   \label{fig:normqq}
\end{figure}
\begin{figure}[tbp]
   \centering
   % This file was created by matlab2tikz. 
%
%The latest updates can be retrieved from
%  http://www.mathworks.com/matlabcentral/fileexchange/22022-matlab2tikz-matlab2tikz
%where you can also make suggestions and rate matlab2tikz.
%
\definecolor{mycolor1}{rgb}{0.00000,0.44700,0.74100}%
%
\begin{tikzpicture}[scale=0.5]

\begin{axis}[%
width=6.028in,
height=4.754in,
at={(1.011in,0.642in)},
scale only axis,
xmin=-4,
xmax=4,
xlabel={\( t_{5} \) Quantiles},
ymin=-4,
ymax=4,
ylabel={\( t_{20} \) Quantiles},
axis background/.style={fill=white},
axis x line*=bottom,
axis y line*=left,
legend style={at={(0.22,0.8)},anchor=south west,legend cell align=left,align=left,draw=white!15!black}
]
\addplot [color=red,solid]%dashdotted,forget plot
  table[row sep=crcr]{%
-3.36492999890722	-3.18227727719706\\
3.36492999890722	3.18227727719706\\
};
\addlegendentry{line passing through 1st and 3rd quartiles};

%% 
 % \addplot [color=red,solid,forget plot]
 %   table[row sep=crcr]{%
 % -0.717907367012227	-0.678938433167201\\
 % 0.717907367012227	0.678938433167201\\
 % };
 %%

\addplot [color=mycolor1,only marks,mark=+,mark options={solid,draw=blue}]%,forget plot
  table[row sep=crcr]{%
-3.36492999890722	-2.52797700274157\\
-2.75650852190947	-2.19665774555268\\
-2.42158470939533	-1.99371260831952\\
-2.19095825718118	-1.84433094160681\\
-2.01504837333302	-1.72471824292079\\
-1.87267775907536	-1.62415118768536\\
-1.75289015787549	-1.5368521116394\\
-1.64930035371341	-1.45934143966002\\
-1.55786934982539	-1.38935650332438\\
-1.47588404882448	-1.32534070698504\\
-1.40143609274796	-1.26617591129329\\
-1.33313328327202	-1.21103094091275\\
-1.26992940170539	-1.15927048168821\\
-1.21101864209398	-1.11039756444688\\
-1.15576734289429	-1.06401577116039\\
-1.10366827295606	-1.01980355411531\\
-1.05430909929714	-0.977496281087634\\
-1.00735007450321	-0.93687336993905\\
-0.962507893023184	-0.897748870600365\\
-0.919543780240827	-0.859964439732386\\
-0.878254550923454	-0.823384012035316\\
-0.838465792122455	-0.78788969777571\\
-0.800026592971978	-0.753378581743777\\
-0.762805418784888	-0.71976019509293\\
-0.726686843800421	-0.686954496448803\\
-0.691568936647646	-0.654890243322396\\
-0.657361147874823	-0.623503666089912\\
-0.623982587858245	-0.592737378986687\\
-0.591360611268586	-0.562539478549754\\
-0.559429644469353	-0.532862791616332\\
-0.528130207043046	-0.503664243611718\\
-0.497408089641652	-0.474904324306311\\
-0.467213658611261	-0.446546633088697\\
-0.437501264092247	-0.41855748951199\\
-0.408228733076414	-0.390905597723995\\
-0.379356932587402	-0.363561755604687\\
-0.35084939101406	-0.336498601163082\\
-0.322671967867415	-0.309690390106078\\
-0.294792563997342	-0.283112799568884\\
-0.267180865704145	-0.256742753854502\\
-0.239808117294835	-0.230558268716293\\
-0.212646917525774	-0.204538311269479\\
-0.185671036089451	-0.178662673062383\\
-0.158855246879451	-0.152911854197701\\
-0.132175175231687	-0.127266956684607\\
-0.105607156713064	-0.101709585436872\\
-0.0791281053272401	-0.0762217555203097\\
-0.0527153892438813	-0.0507858044023847\\
-0.0263467123422732	-0.0253843080737328\\
0	0\\
0.0263467123422732	0.0253843080737328\\
0.0527153892438814	0.0507858044023849\\
0.0791281053272402	0.0762217555203098\\
0.105607156713064	0.101709585436872\\
0.132175175231687	0.127266956684607\\
0.158855246879451	0.152911854197701\\
0.185671036089451	0.178662673062383\\
0.212646917525774	0.204538311269479\\
0.239808117294835	0.230558268716293\\
0.267180865704145	0.256742753854502\\
0.294792563997342	0.283112799568884\\
0.322671967867415	0.309690390106078\\
0.35084939101406	0.336498601163082\\
0.379356932587402	0.363561755604687\\
0.408228733076413	0.390905597723995\\
0.437501264092247	0.41855748951199\\
0.467213658611261	0.446546633088697\\
0.497408089641652	0.474904324306311\\
0.528130207043046	0.503664243611718\\
0.559429644469353	0.532862791616332\\
0.591360611268586	0.562539478549754\\
0.623982587858245	0.592737378986687\\
0.657361147874823	0.623503666089912\\
0.691568936647646	0.654890243322396\\
0.726686843800421	0.686954496448803\\
0.762805418784888	0.71976019509293\\
0.800026592971978	0.753378581743777\\
0.838465792122455	0.78788969777571\\
0.878254550923455	0.823384012035316\\
0.919543780240826	0.859964439732386\\
0.962507893023184	0.897748870600365\\
1.00735007450321	0.93687336993905\\
1.05430909929714	0.977496281087634\\
1.10366827295606	1.01980355411531\\
1.15576734289429	1.06401577116039\\
1.21101864209398	1.11039756444688\\
1.26992940170539	1.15927048168821\\
1.33313328327202	1.21103094091275\\
1.40143609274796	1.26617591129329\\
1.47588404882448	1.32534070698504\\
1.55786934982539	1.38935650332438\\
1.64930035371341	1.45934143966002\\
1.75289015787549	1.5368521116394\\
1.87267775907536	1.62415118768536\\
2.01504837333302	1.72471824292079\\
2.19095825718117	1.84433094160681\\
2.42158470939533	1.99371260831952\\
2.75650852190947	2.19665774555268\\
3.36492999890722	2.52797700274157\\
};
\addlegendentry{Q-Q Plot};

\end{axis}
\end{tikzpicture}%
   \caption{Q-Q Plot: \( t_{20} \) versus \( t_{5} \)}
   \label{fig:t20qq}
\end{figure}
\begin{figure}[tbp]
   \centering
   % This file was created by matlab2tikz. 
%
%The latest updates can be retrieved from
%  http://www.mathworks.com/matlabcentral/fileexchange/22022-matlab2tikz-matlab2tikz
%where you can also make suggestions and rate matlab2tikz.
%
\definecolor{mycolor1}{rgb}{0.00000,0.44700,0.74100}%
%
\begin{tikzpicture}[scale=0.5]

\begin{axis}[%
width=6.028in,
height=4.754in,
at={(1.011in,0.642in)},
scale only axis,
xmin=-4,
xmax=4,
xlabel={\( t_{5} \) Quantiles},
ymin=-4,
ymax=4,
ylabel={\( t_{10} \) Quantiles},
axis background/.style={fill=white},
axis x line*=bottom,
axis y line*=left,
legend style={at={(0.22,0.8)},anchor=south west,legend cell align=left,align=left,draw=white!15!black}
]
\addplot [color=red,solid]%dashdotted,forget plot
  table[row sep=crcr]{%
-3.36492999890722	-3.24140834661461\\
3.36492999890722	3.24140834661461\\
};
\addlegendentry{line passing through 1st and 3rd quartiles};

%% 
 % \addplot [color=red,solid,forget plot]
 %   table[row sep=crcr]{%
 % -0.717907367012227	-0.691554038950371\\
 % 0.717907367012227	0.691554038950371\\
 % };
 %%

\addplot [color=mycolor1,only marks,mark=+,mark options={solid,draw=blue}]%,forget plot
  table[row sep=crcr]{%
-3.36492999890722	-2.7637694581127\\
-2.75650852190947	-2.35931462373653\\
-2.42158470939533	-2.1202335338011\\
-2.19095825718118	-1.9480994688715\\
-2.01504837333302	-1.81246112281168\\
-1.87267775907536	-1.69984087853221\\
-1.75289015787549	-1.60307544542791\\
-1.64930035371341	-1.51789799238776\\
-1.55786934982539	-1.44155945412857\\
-1.47588404882448	-1.37218364111034\\
-1.40143609274796	-1.30843226136284\\
-1.33313328327202	-1.24931669732354\\
-1.26992940170539	-1.19408555534141\\
-1.21101864209398	-1.14215408949299\\
-1.15576734289429	-1.09305807359053\\
-1.10366827295606	-1.04642261049797\\
-1.05430909929714	-1.00194042335574\\
-1.00735007450321	-0.959356365354523\\
-0.962507893023184	-0.918456124463432\\
-0.919543780240827	-0.879057828550588\\
-0.878254550923454	-0.841005699797203\\
-0.838465792122455	-0.804165185198644\\
-0.800026592971978	-0.768419168726629\\
-0.762805418784888	-0.733664988474911\\
-0.726686843800421	-0.699812061312431\\
-0.691568936647646	-0.666779971864189\\
-0.657361147874823	-0.634496920510116\\
-0.623982587858245	-0.60289845192415\\
-0.591360611268586	-0.571926404962191\\
-0.559429644469353	-0.541528038755016\\
-0.528130207043046	-0.511655300219008\\
-0.497408089641652	-0.482264205919681\\
-0.467213658611261	-0.453314317040972\\
-0.437501264092247	-0.4247682906389\\
-0.408228733076414	-0.396591493755621\\
-0.379356932587402	-0.368751669598564\\
-0.35084939101406	-0.341218647040143\\
-0.322671967867415	-0.313964086303864\\
-0.294792563997342	-0.286961254975729\\
-0.267180865704145	-0.26018482949208\\
-0.239808117294835	-0.233610718063889\\
-0.212646917525774	-0.207215901646676\\
-0.185671036089451	-0.180978290087965\\
-0.158855246879451	-0.154876591005921\\
-0.132175175231687	-0.128890189293274\\
-0.105607156713064	-0.102999035414941\\
-0.0791281053272401	-0.0771835408877566\\
-0.0527153892438813	-0.0514244795054024\\
-0.0263467123422732	-0.0257028930080903\\
0	0\\
0.0263467123422732	0.0257028930080903\\
0.0527153892438814	0.0514244795054025\\
0.0791281053272402	0.0771835408877568\\
0.105607156713064	0.102999035414941\\
0.132175175231687	0.128890189293274\\
0.158855246879451	0.154876591005921\\
0.185671036089451	0.180978290087965\\
0.212646917525774	0.207215901646676\\
0.239808117294835	0.233610718063889\\
0.267180865704145	0.26018482949208\\
0.294792563997342	0.286961254975729\\
0.322671967867415	0.313964086303864\\
0.35084939101406	0.341218647040143\\
0.379356932587402	0.368751669598564\\
0.408228733076413	0.396591493755621\\
0.437501264092247	0.4247682906389\\
0.467213658611261	0.453314317040971\\
0.497408089641652	0.482264205919681\\
0.528130207043046	0.511655300219008\\
0.559429644469353	0.541528038755016\\
0.591360611268586	0.571926404962191\\
0.623982587858245	0.60289845192415\\
0.657361147874823	0.634496920510116\\
0.691568936647646	0.666779971864189\\
0.726686843800421	0.699812061312431\\
0.762805418784888	0.733664988474911\\
0.800026592971978	0.768419168726629\\
0.838465792122455	0.804165185198644\\
0.878254550923455	0.841005699797203\\
0.919543780240826	0.87905782855059\\
0.962507893023184	0.918456124463431\\
1.00735007450321	0.959356365354523\\
1.05430909929714	1.00194042335574\\
1.10366827295606	1.04642261049797\\
1.15576734289429	1.09305807359053\\
1.21101864209398	1.14215408949299\\
1.26992940170539	1.19408555534141\\
1.33313328327202	1.24931669732354\\
1.40143609274796	1.30843226136284\\
1.47588404882448	1.37218364111034\\
1.55786934982539	1.44155945412857\\
1.64930035371341	1.51789799238776\\
1.75289015787549	1.60307544542791\\
1.87267775907536	1.69984087853221\\
2.01504837333302	1.81246112281167\\
2.19095825718117	1.9480994688715\\
2.42158470939533	2.1202335338011\\
2.75650852190947	2.35931462373653\\
3.36492999890722	2.7637694581127\\
};
\addlegendentry{Q-Q Plot};

\end{axis}
\end{tikzpicture}%
   \caption{Q-Q Plot: \( t_{10} \) versus \( t_{5} \)}
   \label{fig:t10qq}
\end{figure}
\begin{figure}[tbp]
   \centering
   % This file was created by matlab2tikz. 
%
%The latest updates can be retrieved from
%  http://www.mathworks.com/matlabcentral/fileexchange/22022-matlab2tikz-matlab2tikz
%where you can also make suggestions and rate matlab2tikz.
%
\definecolor{mycolor1}{rgb}{0.00000,0.44700,0.74100}%
%
\begin{tikzpicture}[scale=0.5]

\begin{axis}[%
width=6.028in,
height=4.754in,
at={(1.011in,0.642in)},
scale only axis,
xmin=-4,
xmax=4,
xlabel={\( t_{5} \) Quantiles},
ymin=-4,
ymax=4,
ylabel={\( t_{5} \) Quantiles},
axis background/.style={fill=white},
axis x line*=bottom,
axis y line*=left,
legend style={at={(0.22,0.8)},anchor=south west,legend cell align=left,align=left,draw=white!15!black}
]
\addplot [color=red,solid]%dashdotted,forget plot
  table[row sep=crcr]{%
-3.36492999890722	-3.36492999890722\\
3.36492999890722	3.36492999890722\\
};
\addlegendentry{line passing through 1st and 3rd quartiles};

%% 
 % \addplot [color=red,solid,forget plot]
 %   table[row sep=crcr]{%
 % -0.717907367012227    -0.717907367012227\\
 % 0.717907367012227    0.717907367012227\\
 % };
 %%

\addplot [color=mycolor1,only marks,mark=+,mark options={solid,draw=blue}]%,forget plot
  table[row sep=crcr]{%
-3.36492999890722	-3.36492999890722\\
-2.75650852190947	-2.75650852190947\\
-2.42158470939533	-2.42158470939533\\
-2.19095825718118	-2.19095825718118\\
-2.01504837333302	-2.01504837333302\\
-1.87267775907536	-1.87267775907536\\
-1.75289015787549	-1.75289015787549\\
-1.64930035371341	-1.64930035371341\\
-1.55786934982539	-1.55786934982539\\
-1.47588404882448	-1.47588404882448\\
-1.40143609274796	-1.40143609274796\\
-1.33313328327202	-1.33313328327202\\
-1.26992940170539	-1.26992940170539\\
-1.21101864209398	-1.21101864209398\\
-1.15576734289429	-1.15576734289429\\
-1.10366827295606	-1.10366827295606\\
-1.05430909929714	-1.05430909929714\\
-1.00735007450321	-1.00735007450321\\
-0.962507893023184	-0.962507893023184\\
-0.919543780240827	-0.919543780240827\\
-0.878254550923454	-0.878254550923454\\
-0.838465792122455	-0.838465792122455\\
-0.800026592971978	-0.800026592971978\\
-0.762805418784888	-0.762805418784888\\
-0.726686843800421	-0.726686843800421\\
-0.691568936647646	-0.691568936647646\\
-0.657361147874823	-0.657361147874823\\
-0.623982587858245	-0.623982587858245\\
-0.591360611268586	-0.591360611268586\\
-0.559429644469353	-0.559429644469353\\
-0.528130207043046	-0.528130207043046\\
-0.497408089641652	-0.497408089641652\\
-0.467213658611261	-0.467213658611261\\
-0.437501264092247	-0.437501264092247\\
-0.408228733076414	-0.408228733076414\\
-0.379356932587402	-0.379356932587402\\
-0.35084939101406	-0.35084939101406\\
-0.322671967867415	-0.322671967867415\\
-0.294792563997342	-0.294792563997342\\
-0.267180865704145	-0.267180865704145\\
-0.239808117294835	-0.239808117294835\\
-0.212646917525774	-0.212646917525774\\
-0.185671036089451	-0.185671036089451\\
-0.158855246879451	-0.158855246879451\\
-0.132175175231687	-0.132175175231687\\
-0.105607156713064	-0.105607156713064\\
-0.0791281053272401	-0.0791281053272401\\
-0.0527153892438813	-0.0527153892438813\\
-0.0263467123422732	-0.0263467123422732\\
0	0\\
0.0263467123422732	0.0263467123422732\\
0.0527153892438814	0.0527153892438814\\
0.0791281053272402	0.0791281053272402\\
0.105607156713064	0.105607156713064\\
0.132175175231687	0.132175175231687\\
0.158855246879451	0.158855246879451\\
0.185671036089451	0.185671036089451\\
0.212646917525774	0.212646917525774\\
0.239808117294835	0.239808117294835\\
0.267180865704145	0.267180865704145\\
0.294792563997342	0.294792563997342\\
0.322671967867415	0.322671967867415\\
0.35084939101406	0.35084939101406\\
0.379356932587402	0.379356932587402\\
0.408228733076413	0.408228733076413\\
0.437501264092247	0.437501264092247\\
0.467213658611261	0.467213658611261\\
0.497408089641652	0.497408089641652\\
0.528130207043046	0.528130207043046\\
0.559429644469353	0.559429644469353\\
0.591360611268586	0.591360611268586\\
0.623982587858245	0.623982587858245\\
0.657361147874823	0.657361147874823\\
0.691568936647646	0.691568936647646\\
0.726686843800421	0.726686843800421\\
0.762805418784888	0.762805418784888\\
0.800026592971978	0.800026592971978\\
0.838465792122455	0.838465792122455\\
0.878254550923455	0.878254550923455\\
0.919543780240826	0.919543780240826\\
0.962507893023184	0.962507893023184\\
1.00735007450321	1.00735007450321\\
1.05430909929714	1.05430909929714\\
1.10366827295606	1.10366827295606\\
1.15576734289429	1.15576734289429\\
1.21101864209398	1.21101864209398\\
1.26992940170539	1.26992940170539\\
1.33313328327202	1.33313328327202\\
1.40143609274796	1.40143609274796\\
1.47588404882448	1.47588404882448\\
1.55786934982539	1.55786934982539\\
1.64930035371341	1.64930035371341\\
1.75289015787549	1.75289015787549\\
1.87267775907536	1.87267775907536\\
2.01504837333302	2.01504837333302\\
2.19095825718117	2.19095825718117\\
2.42158470939533	2.42158470939533\\
2.75650852190947	2.75650852190947\\
3.36492999890722	3.36492999890722\\
};
\addlegendentry{Q-Q Plot};

\end{axis}
\end{tikzpicture}%
   \caption{Q-Q Plot: \( t_{5} \) versus \( t_{5} \)}
   \label{fig:t5qq}
\end{figure}
\begin{figure}[tbp]
   \centering
   % This file was created by matlab2tikz. 
%
%The latest updates can be retrieved from
%  http://www.mathworks.com/matlabcentral/fileexchange/22022-matlab2tikz-matlab2tikz
%where you can also make suggestions and rate matlab2tikz.
%
\definecolor{mycolor1}{rgb}{0.00000,0.44700,0.74100}%
%
\begin{tikzpicture}[scale=0.5]

\begin{axis}[%
width=6.028in,
height=4.754in,
at={(1.011in,0.642in)},
scale only axis,
xmin=-4,
xmax=4,
xlabel={\( t_{5} \) Quantiles},
ymin=-8,
ymax=8,
ylabel={\( t_{2} \) Quantiles},
axis background/.style={fill=white},
axis x line*=bottom,
axis y line*=left,
legend style={at={(0.22,0.8)},anchor=south west,legend cell align=left,align=left,draw=white!15!black}
]
\addplot [color=red,solid]%dashdotted,forget plot
  table[row sep=crcr]{
-3.36492999890722	-3.77698946757965\\
3.36492999890722	3.77698946757965\\
};
\addlegendentry{line passing through 1st and 3rd quartiles};

%% 
 % \addplot [color=red,solid,forget plot]
 %   table[row sep=crcr]{%
 % -0.717907367012227	-0.805820199761543\\
 % 0.717907367012227	0.805820199761543\\
 % };
 %%

\addplot [color=mycolor1,only marks,mark=+,mark options={solid,draw=blue}]%,forget plot
  table[row sep=crcr]{
-3.36492999890722	-6.96455673428323\\
-2.75650852190947	-4.84873221385061\\
-2.42158470939533	-3.89642535976149\\
-2.19095825718118	-3.31976404784035\\
-2.01504837333302	-2.91998558035373\\
-1.87267775907536	-2.62016187037182\\
-1.75289015787549	-2.38337818053688\\
-1.64930035371341	-2.18940134759929\\
-1.55786934982539	-2.02608146554034\\
-1.47588404882448	-1.88561808316413\\
-1.40143609274796	-1.76273999735172\\
-1.33313328327202	-1.65373773157913\\
-1.26992940170539	-1.55591339406537\\
-1.21101864209398	-1.4672504626906\\
-1.15576734289429	-1.38620656016734\\
-1.10366827295606	-1.31157847467778\\
-1.05430909929714	-1.24241140897146\\
-1.00735007450321	-1.1779362781317\\
-0.962507893023184	-1.11752534556747\\
-0.919543780240827	-1.06066017177982\\
-0.878254550923454	-1.00690802626528\\
-0.838465792122455	-0.955904238798486\\
-0.800026592971978	-0.907338797158432\\
-0.762805418784888	-0.860946032092278\\
-0.726686843800421	-0.816496580927726\\
-0.691568936647646	-0.773791056262993\\
-0.657361147874823	-0.732655006603489\\
-0.623982587858245	-0.692934867183583\\
-0.591360611268586	-0.654494677707282\\
-0.559429644469353	-0.617213399848368\\
-0.528130207043046	-0.580982707974908\\
-0.497408089641652	-0.545705156331749\\
-0.467213658611261	-0.511292647969764\\
-0.437501264092247	-0.477665147221226\\
-0.408228733076414	-0.444749589996661\\
-0.379356932587402	-0.412478955692153\\
-0.35084939101406	-0.38079147180858\\
-0.322671967867415	-0.349629928048839\\
-0.294792563997342	-0.318941081077404\\
-0.267180865704145	-0.288675134594813\\
-0.239808117294835	-0.258785282117425\\
-0.212646917525774	-0.229227302024501\\
-0.185671036089451	-0.199959196164015\\
-0.158855246879451	-0.170940864689457\\
-0.132175175231687	-0.14213381090374\\
-0.105607156713064	-0.113500870767833\\
-0.0791281053272401	-0.0850059624328604\\
-0.0527153892438813	-0.0566138517072297\\
-0.0263467123422732	-0.0282899297993336\\
0	0\\
0.0263467123422732	0.0282899297993336\\
0.0527153892438814	0.0566138517072298\\
0.0791281053272402	0.0850059624328606\\
0.105607156713064	0.113500870767833\\
0.132175175231687	0.14213381090374\\
0.158855246879451	0.170940864689457\\
0.185671036089451	0.199959196164015\\
0.212646917525774	0.229227302024501\\
0.239808117294835	0.258785282117425\\
0.267180865704145	0.288675134594813\\
0.294792563997342	0.318941081077404\\
0.322671967867415	0.349629928048839\\
0.35084939101406	0.38079147180858\\
0.379356932587402	0.412478955692152\\
0.408228733076413	0.44474958999666\\
0.437501264092247	0.477665147221225\\
0.467213658611261	0.511292647969764\\
0.497408089641652	0.545705156331749\\
0.528130207043046	0.580982707974908\\
0.559429644469353	0.617213399848367\\
0.591360611268586	0.654494677707282\\
0.623982587858245	0.692934867183583\\
0.657361147874823	0.732655006603489\\
0.691568936647646	0.773791056262993\\
0.726686843800421	0.816496580927726\\
0.762805418784888	0.860946032092278\\
0.800026592971978	0.907338797158432\\
0.838465792122455	0.955904238798486\\
0.878254550923455	1.00690802626528\\
0.919543780240826	1.06066017177982\\
0.962507893023184	1.11752534556747\\
1.00735007450321	1.1779362781317\\
1.05430909929714	1.24241140897146\\
1.10366827295606	1.31157847467778\\
1.15576734289429	1.38620656016734\\
1.21101864209398	1.4672504626906\\
1.26992940170539	1.55591339406537\\
1.33313328327202	1.65373773157913\\
1.40143609274796	1.76273999735172\\
1.47588404882448	1.88561808316413\\
1.55786934982539	2.02608146554034\\
1.64930035371341	2.18940134759928\\
1.75289015787549	2.38337818053688\\
1.87267775907536	2.62016187037182\\
2.01504837333302	2.91998558035372\\
2.19095825718117	3.31976404784035\\
2.42158470939533	3.89642535976149\\
2.75650852190947	4.84873221385061\\
3.36492999890722	6.96455673428323\\
};
\addlegendentry{Q-Q Plot};

\end{axis}
\end{tikzpicture}%
   \caption{Q-Q Plot: \( t_{2} \) versus \( t_{5} \)}
   \label{fig:t2qq}
\end{figure}
\begin{figure}[tbp]
   \centering
   % This file was created by matlab2tikz. 
%
%The latest updates can be retrieved from
%  http://www.mathworks.com/matlabcentral/fileexchange/22022-matlab2tikz-matlab2tikz
%where you can also make suggestions and rate matlab2tikz.
%
\definecolor{mycolor1}{rgb}{0.00000,0.44700,0.74100}%
%
\begin{tikzpicture}[scale=0.5]

\begin{axis}[%
width=6.028in,
height=4.754in,
at={(1.011in,0.642in)},
scale only axis,
xmin=-4,
xmax=4,
xlabel={\( t_{5} \) Quantiles},
ymin=-40,
ymax=40,
ylabel={\( t_{1} \) (or Cauchy) Quantiles},
axis background/.style={fill=white},
axis x line*=bottom,
axis y line*=left,
legend style={at={(0.22,0.8)},anchor=south west,legend cell align=left,align=left,draw=white!15!black}
]
\addplot [color=red,solid]%dashdotted,forget plot
  table[row sep=crcr]{%
-3.36492999890722	-4.61573115225121\\
3.36492999890722	4.61573115225121\\
};
\addlegendentry{line passing through 1st and 3rd quartiles};

%% 
 % \addplot [color=red,solid,forget plot]
 %   table[row sep=crcr]{%
 % -0.717907367012227	-0.984765626454373\\
 % 0.717907367012227	0.984765626454373\\
 % };
 %%

\addplot [color=mycolor1,only marks,mark=+,mark options={solid,draw=blue}]%,forget plot
  table[row sep=crcr]{%
-3.36492999890722	-31.8205159537739\\
-2.75650852190947	-15.8945448438653\\
-2.42158470939533	-10.5788949934056\\
-2.19095825718118	-7.91581508830583\\
-2.01504837333302	-6.31375151467504\\
-1.87267775907536	-5.24218358111317\\
-1.75289015787549	-4.47374282921155\\
-1.64930035371341	-3.89474285492986\\
-1.55786934982539	-3.44202257666922\\
-1.47588404882448	-3.07768353717525\\
-1.40143609274796	-2.77760685391497\\
-1.33313328327202	-2.5257116894473\\
-1.26992940170539	-2.31086365388241\\
-1.21101864209398	-2.1251081731572\\
-1.15576734289429	-1.96261050550515\\
-1.10366827295606	-1.81899324728107\\
-1.05430909929714	-1.69090765578501\\
-1.00735007450321	-1.57574785996865\\
-0.962507893023184	-1.47145531581997\\
-0.919543780240827	-1.37638192047117\\
-0.878254550923454	-1.28919223178507\\
-0.838465792122455	-1.20879235040961\\
-0.800026592971978	-1.13427734925541\\
-0.762805418784888	-1.06489184032479\\
-0.726686843800421	-1\\
-0.691568936647646	-0.939062505817492\\
-0.657361147874823	-0.881618592363189\\
-0.623982587858245	-0.827271945972475\\
-0.591360611268586	-0.775679511049613\\
-0.559429644469353	-0.726542528005361\\
-0.528130207043046	-0.679599298224526\\
-0.497408089641652	-0.634619297544148\\
-0.467213658611261	-0.591398351399471\\
-0.437501264092247	-0.54975465219277\\
-0.408228733076414	-0.509525449494429\\
-0.379356932587402	-0.470564281212251\\
-0.35084939101406	-0.432738642247426\\
-0.322671967867415	-0.395928008797721\\
-0.294792563997342	-0.360022153095757\\
-0.267180865704145	-0.324919696232906\\
-0.239808117294835	-0.290526856731916\\
-0.212646917525774	-0.256756360367727\\
-0.185671036089451	-0.223526482897149\\
-0.158855246879451	-0.190760202218567\\
-0.132175175231687	-0.158384440324536\\
-0.105607156713064	-0.126329378446108\\
-0.0791281053272401	-0.094527831179282\\
-0.0527153892438813	-0.0629146672536496\\
-0.0263467123422732	-0.0314262660433512\\
0	0\\
0.0263467123422732	0.0314262660433512\\
0.0527153892438814	0.0629146672536498\\
0.0791281053272402	0.0945278311792821\\
0.105607156713064	0.126329378446108\\
0.132175175231687	0.158384440324536\\
0.158855246879451	0.190760202218567\\
0.185671036089451	0.223526482897149\\
0.212646917525774	0.256756360367727\\
0.239808117294835	0.290526856731916\\
0.267180865704145	0.324919696232906\\
0.294792563997342	0.360022153095757\\
0.322671967867415	0.395928008797721\\
0.35084939101406	0.432738642247426\\
0.379356932587402	0.470564281212251\\
0.408228733076413	0.509525449494428\\
0.437501264092247	0.54975465219277\\
0.467213658611261	0.591398351399471\\
0.497408089641652	0.634619297544148\\
0.528130207043046	0.679599298224526\\
0.559429644469353	0.726542528005361\\
0.591360611268586	0.775679511049613\\
0.623982587858245	0.827271945972475\\
0.657361147874823	0.881618592363189\\
0.691568936647646	0.939062505817492\\
0.726686843800421	1\\
0.762805418784888	1.06489184032479\\
0.800026592971978	1.13427734925541\\
0.838465792122455	1.20879235040961\\
0.878254550923455	1.28919223178507\\
0.919543780240826	1.37638192047117\\
0.962507893023184	1.47145531581997\\
1.00735007450321	1.57574785996865\\
1.05430909929714	1.69090765578501\\
1.10366827295606	1.81899324728107\\
1.15576734289429	1.96261050550515\\
1.21101864209398	2.1251081731572\\
1.26992940170539	2.31086365388241\\
1.33313328327202	2.5257116894473\\
1.40143609274796	2.77760685391497\\
1.47588404882448	3.07768353717525\\
1.55786934982539	3.44202257666922\\
1.64930035371341	3.89474285492986\\
1.75289015787549	4.47374282921155\\
1.87267775907536	5.24218358111317\\
2.01504837333302	6.31375151467503\\
2.19095825718117	7.91581508830581\\
2.42158470939533	10.5788949934056\\
2.75650852190947	15.8945448438653\\
3.36492999890722	31.8205159537739\\
};
\addlegendentry{Q-Q Plot};

\end{axis}
\end{tikzpicture}%
   \caption{Q-Q Plot: \( t_{1} \) (or Cauchy) versus \( t_{5} \)}
   \label{fig:t1qq}
\end{figure}
\begin{figure}[tbp]
   \centering
   % This file was created by matlab2tikz. 
%
%The latest updates can be retrieved from
%  http://www.mathworks.com/matlabcentral/fileexchange/22022-matlab2tikz-matlab2tikz
%where you can also make suggestions and rate matlab2tikz.
%
\definecolor{mycolor1}{rgb}{0.00000,0.44700,0.74100}%
%
\begin{tikzpicture}[scale=0.5]

\begin{axis}[%
width=6.028in,
height=4.754in,
at={(1.011in,0.642in)},
scale only axis,
xmin=-4,
xmax=4,
xlabel={\( t_{5} \) Quantiles},
ymin=-8,
ymax=10,
ylabel={\( (\chi^{2}_{30}-30) \) Quantiles},
axis background/.style={fill=white},
axis x line*=bottom,
axis y line*=left,
legend style={at={(0.22,0.8)},anchor=south west,legend cell align=left,align=left,draw=white!15!black}
]
\addplot [color=red,solid]%dashdotted,forget plot
  table[row sep=crcr]{%
-3.36492999890722	-7.05269526413488\\
3.36492999890722	6.36034016084415\\
};
\addlegendentry{line passing through 1st and 3rd quartiles};
%% 
 % \addplot [color=red,solid,forget plot]
 %   table[row sep=crcr]{%
 % -0.717907367012227	-1.77701221220526\\
 % 0.717907367012227	1.08465710891453\\
 % };
 %%
\addplot [color=mycolor1,only marks,mark=+,mark options={solid,draw=blue}]%,forget plot
  table[row sep=crcr]{%
-3.36492999890722	-2.88516819810088\\
-2.75650852190947	-2.81516818097134\\
-2.42158470939533	-2.7549012912897\\
-2.19095825718118	-2.69984858120277\\
-2.01504837333302	-2.64815368225073\\
-1.87267775907536	-2.59882664199938\\
-1.75289015787549	-2.55126129867303\\
-1.64930035371341	-2.50505243830279\\
-1.55786934982539	-2.45991200707444\\
-1.47588404882448	-2.41562562584482\\
-1.40143609274796	-2.3720279248744\\
-1.33313328327202	-2.32898758433771\\
-1.26992940170539	-2.28639777221243\\
-1.21101864209398	-2.24416976745818\\
-1.15576734289429	-2.20222855575528\\
-1.10366827295606	-2.16050969646626\\
-1.05430909929714	-2.11895703679149\\
-1.00735007450321	-2.0775210069222\\
-0.962507893023184	-2.03615732359388\\
-0.919543780240827	-1.99482598694765\\
-0.878254550923454	-1.95349049203959\\
-0.838465792122455	-1.9121172000511\\
-0.800026592971978	-1.87067483006205\\
-0.762805418784888	-1.82913404301291\\
-0.726686843800421	-1.78746709695433\\
-0.691568936647646	-1.74564755795804\\
-0.657361147874823	-1.70365005484647\\
-0.623982587858245	-1.66145006864941\\
-0.591360611268586	-1.6190237497217\\
-0.559429644469353	-1.57634775696472\\
-0.528130207043046	-1.53339911472878\\
-0.497408089641652	-1.49015508383435\\
-0.467213658611261	-1.44659304380774\\
-0.437501264092247	-1.40269038393171\\
-0.408228733076414	-1.35842440110118\\
-0.379356932587402	-1.31377220277408\\
-0.35084939101406	-1.26871061353836\\
-0.322671967867415	-1.22321608399224\\
-0.294792563997342	-1.1772646007662\\
-0.267180865704145	-1.13083159661128\\
-0.239808117294835	-1.08389185954356\\
-0.212646917525774	-1.0364194400741\\
-0.185671036089451	-0.988387555570398\\
-0.158855246879451	-0.939768490790561\\
-0.132175175231687	-0.890533493607214\\
-0.105607156713064	-0.840652664894154\\
-0.0791281053272401	-0.790094841484736\\
-0.0527153892438813	-0.738827471026338\\
-0.0263467123422732	-0.686816477447817\\
0	-0.634026115624662\\
0.0263467123422732	-0.580418813666108\\
0.0527153892438814	-0.52595500105593\\
0.0791281053272402	-0.470592920648726\\
0.105607156713064	-0.41428842224984\\
0.132175175231687	-0.356994735181745\\
0.158855246879451	-0.298662216852698\\
0.185671036089451	-0.239238073882775\\
0.212646917525774	-0.178666051792556\\
0.239808117294835	-0.116886088602039\\
0.267180865704145	-0.0538339268980481\\
0.294792563997342	0.0105593220221571\\
0.322671967867415	0.076367669472603\\
0.35084939101406	0.143670805102432\\
0.379356932587402	0.21255472766055\\
0.408228733076413	0.2831124635255\\
0.437501264092247	0.355444888294746\\
0.467213658611261	0.429661669910782\\
0.497408089641652	0.505882355768182\\
0.528130207043046	0.584237631218944\\
0.559429644469353	0.664870783170318\\
0.591360611268586	0.747939410446757\\
0.623982587858245	0.833617432802338\\
0.657361147874823	0.922097463653413\\
0.691568936647646	1.01359362876118\\
0.726686843800421	1.10834493563232\\
0.762805418784888	1.20661932829835\\
0.800026592971978	1.3087186021953\\
0.838465792122455	1.41498440816162\\
0.878254550923455	1.52580564906147\\
0.919543780240826	1.64162767608745\\
0.962507893023184	1.76296383781555\\
1.00735007450321	1.89041014424292\\
1.05430909929714	2.02466411282651\\
1.10366827295606	2.16654931616078\\
1.15576734289429	2.3170478373171\\
1.21101864209398	2.47734390432258\\
1.26992940170539	2.64888367296753\\
1.33313328327202	2.83345891340976\\
1.40143609274796	3.03332708539163\\
1.47588404882448	3.25138863117032\\
1.55786934982539	3.49145771584493\\
1.64930035371341	3.75869261969684\\
1.75289015787549	4.06031416781787\\
1.87267775907536	4.40688004310323\\
2.01504837333302	4.81472790325118\\
2.19095825718117	5.31117091082631\\
2.42158470939533	5.94728749887945\\
2.75650852190947	6.83740931119259\\
3.36492999890722	8.34486673014437\\
};
\addlegendentry{Q-Q Plot};

\end{axis}
\end{tikzpicture}%
   \caption{Q-Q Plot: \( (\chi^{2}_{3}-3) \) versus \( t_{5} \)}
   \label{fig:chisq3qq}
\end{figure}
\begin{figure}[tbp]
   \centering
   % This file was created by matlab2tikz. 
%
%The latest updates can be retrieved from
%  http://www.mathworks.com/matlabcentral/fileexchange/22022-matlab2tikz-matlab2tikz
%where you can also make suggestions and rate matlab2tikz.
%
\definecolor{mycolor1}{rgb}{0.00000,0.44700,0.74100}%
%
\begin{tikzpicture}[scale=0.5]

\begin{axis}[%
width=6.028in,
height=4.754in,
at={(1.011in,0.642in)},
scale only axis,
xmin=-4,
xmax=4,
xlabel={\( t_{5} \) Quantiles},
ymin=-25,
ymax=25,
ylabel={\( (\chi^{2}_{30}-30) \) Quantiles},
axis background/.style={fill=white},
axis x line*=bottom,
axis y line*=left,
legend style={at={(0.22,0.8)},anchor=south west,legend cell align=left,align=left,draw=white!15!black}
]
\addplot [color=red,solid]
  table[row sep=crcr]{
-3.36492999890722	-24.2790776659813\\
3.36492999890722	23.5427802955136\\
};
\addlegendentry{line passing through 1st and 3rd quartiles};

%% 
 % \addplot [color=red,solid]
 %   table[row sep=crcr]{%
 % -0.717907367012227	-5.46954219801407\\
 % 0.717907367012227	4.73324482754634\\
 % };
 % \addlegendentry{data2};
 %%

\addplot [color=mycolor1,only marks,mark=+,mark options={solid,draw=blue}]
  table[row sep=crcr]{
-3.36492999890722	-15.0465434715446\\
-2.75650852190947	-13.6938252832708\\
-2.42158470939533	-12.7923753180451\\
-2.19095825718118	-12.0917214502449\\
-2.01504837333302	-11.5073390180465\\
-1.87267775907536	-10.9996173570955\\
-1.75289015787549	-10.5465700427175\\
-1.64930035371341	-10.1346316502224\\
-1.55786934982539	-9.75479853966634\\
-1.47588404882448	-9.40076538541466\\
-1.40143609274796	-9.06793069570904\\
-1.33313328327202	-8.75282468814286\\
-1.26992940170539	-8.45276025851469\\
-1.21101864209398	-8.16560976075052\\
-1.15576734289429	-7.88965659455114\\
-1.10366827295606	-7.62349326497822\\
-1.05430909929714	-7.36594939505746\\
-1.00735007450321	-7.11603966148991\\
-0.962507893023184	-6.87292534747001\\
-0.919543780240827	-6.63588542620993\\
-0.878254550923454	-6.40429445605071\\
-0.838465792122455	-6.17760543500624\\
-0.800026592971978	-5.95533632657022\\
-0.762805418784888	-5.73705934397809\\
-0.726686843800421	-5.52239233511374\\
-0.691568936647646	-5.31099178671507\\
-0.657361147874823	-5.1025470907181\\
-0.623982587858245	-4.89677580431991\\
-0.591360611268586	-4.69341969964729\\
-0.559429644469353	-4.49224144611971\\
-0.528130207043046	-4.29302180365332\\
-0.497408089641652	-4.09555723117703\\
-0.467213658611261	-3.89965783489636\\
-0.437501264092247	-3.7051455960241\\
-0.408228733076414	-3.51185282949578\\
-0.379356932587402	-3.31962083436888\\
-0.35084939101406	-3.12829870379863\\
-0.322671967867415	-2.93774226815698\\
-0.294792563997342	-2.74781314935975\\
-0.267180865704145	-2.55837790805107\\
-0.239808117294835	-2.36930726815896\\
-0.212646917525774	-2.18047540563126\\
-0.185671036089451	-1.99175928999995\\
-0.158855246879451	-1.80303806889265\\
-0.132175175231687	-1.61419248677889\\
-0.105607156713064	-1.42510433015813\\
-0.0791281053272401	-1.23565589210513\\
-0.0527153892438813	-1.04572944961614\\
-0.0263467123422732	-0.855206747569309\\
0	-0.663968483338419\\
0.0263467123422732	-0.471893786192727\\
0.0527153892438814	-0.278859685580823\\
0.0791281053272402	-0.0847405622341455\\
0.105607156713064	0.110592424267871\\
0.132175175231687	0.307271938262435\\
0.158855246879451	0.505435108477005\\
0.185671036089451	0.705224214350796\\
0.212646917525774	0.906787427302984\\
0.239808117294835	1.11027961723669\\
0.267180865704145	1.31586323603909\\
0.294792563997342	1.52370929163506\\
0.322671967867415	1.73399842835942\\
0.35084939101406	1.94692213211055\\
0.379356932587402	2.16268408206003\\
0.408228733076413	2.38150167476578\\
0.437501264092247	2.60360775156488\\
0.467213658611261	2.82925256635737\\
0.497408089641652	3.0587060386646\\
0.528130207043046	3.29226034658569\\
0.559429644469353	3.53023292655934\\
0.591360611268586	3.7729699624316\\
0.623982587858245	4.02085046626951\\
0.657361147874823	4.27429107905315\\
0.691568936647646	4.53375175276256\\
0.726686843800421	4.79974251914093\\
0.762805418784888	5.07283160834442\\
0.800026592971978	5.35365525816028\\
0.838465792122455	5.64292965924415\\
0.878254550923455	5.94146562523191\\
0.919543780240826	6.25018677545154\\
0.962507893023184	6.57015229770559\\
1.00735007450321	6.90258575827021\\
1.05430909929714	7.24891200710412\\
1.10366827295606	7.61080508642258\\
1.15576734289429	7.99025135138281\\
1.21101864209398	8.38963402430618\\
1.26992940170539	8.81184860056553\\
1.33313328327202	9.26046375287783\\
1.40143609274796	9.73995122376564\\
1.47588404882448	10.2560237387118\\
1.55786934982539	10.8161485419585\\
1.64930035371341	11.4303595190604\\
1.75289015787549	12.1126051681284\\
1.87267775907536	12.8831246942893\\
2.01504837333302	13.7729718257422\\
2.19095825718117	14.8335589499889\\
2.42158470939533	16.1599326982613\\
2.75650852190947	17.9618028179859\\
3.36492999890722	20.8921813115171\\
};
\addlegendentry{Q-Q Plot};

\end{axis}
\end{tikzpicture}%
   \caption{Q-Q Plot: \( (\chi^{2}_{30}-30) \) versus \( t_{5} \)}
   \label{fig:chisq30qq}
\end{figure}
\begin{figure}[tbp]
   \centering
   % This file was created by matlab2tikz. 
%
%The latest updates can be retrieved from
%  http://www.mathworks.com/matlabcentral/fileexchange/22022-matlab2tikz-matlab2tikz
%where you can also make suggestions and rate matlab2tikz.
%
\definecolor{mycolor1}{rgb}{0.00000,0.44700,0.74100}%
%
\begin{tikzpicture}[scale=0.5]

\begin{axis}[%
width=6.028in,
height=4.754in,
at={(1.011in,0.642in)},
scale only axis,
xmin=-4,
xmax=4,
xlabel={\( t_{5} \) Quantiles},
ymin=6,
ymax=14,
ylabel={\( (t_{5}+10) \) Quantiles},
axis background/.style={fill=white},
axis x line*=bottom,
axis y line*=left,
legend style={at={(0.22,0.8)},anchor=south west,legend cell align=left,align=left,draw=white!15!black}
]
\addplot [color=red,solid]%dashdotted,forget plot
  table[row sep=crcr]{%
-3.36492999890722	6.63507000109278\\
3.36492999890722	13.3649299989072\\
};
\addlegendentry{line passing through 1st and 3rd quartiles};

%% 
 % \addplot [color=red,solid,forget plot]
 %   table[row sep=crcr]{%
 % -0.717907367012227    9.28209263298777\\
 % 0.717907367012227    10.7179073670122\\
 % };
 %%

\addplot [color=mycolor1,only marks,mark=+,mark options={solid,draw=blue}]%,forget plot
  table[row sep=crcr]{%%
-3.36492999890722	6.63507000109278\\
-2.75650852190947	7.24349147809053\\
-2.42158470939533	7.57841529060467\\
-2.19095825718118	7.80904174281882\\
-2.01504837333302	7.98495162666698\\
-1.87267775907536	8.12732224092464\\
-1.75289015787549	8.24710984212451\\
-1.64930035371341	8.35069964628659\\
-1.55786934982539	8.44213065017461\\
-1.47588404882448	8.52411595117552\\
-1.40143609274796	8.59856390725204\\
-1.33313328327202	8.66686671672798\\
-1.26992940170539	8.73007059829461\\
-1.21101864209398	8.78898135790602\\
-1.15576734289429	8.84423265710571\\
-1.10366827295606	8.89633172704394\\
-1.05430909929714	8.94569090070286\\
-1.00735007450321	8.99264992549679\\
-0.962507893023184	9.03749210697682\\
-0.919543780240827	9.08045621975917\\
-0.878254550923454	9.12174544907655\\
-0.838465792122455	9.16153420787754\\
-0.800026592971978	9.19997340702802\\
-0.762805418784888	9.23719458121511\\
-0.726686843800421	9.27331315619958\\
-0.691568936647646	9.30843106335236\\
-0.657361147874823	9.34263885212518\\
-0.623982587858245	9.37601741214175\\
-0.591360611268586	9.40863938873141\\
-0.559429644469353	9.44057035553065\\
-0.528130207043046	9.47186979295695\\
-0.497408089641652	9.50259191035835\\
-0.467213658611261	9.53278634138874\\
-0.437501264092247	9.56249873590775\\
-0.408228733076414	9.59177126692359\\
-0.379356932587402	9.6206430674126\\
-0.35084939101406	9.64915060898594\\
-0.322671967867415	9.67732803213258\\
-0.294792563997342	9.70520743600266\\
-0.267180865704145	9.73281913429586\\
-0.239808117294835	9.76019188270516\\
-0.212646917525774	9.78735308247423\\
-0.185671036089451	9.81432896391055\\
-0.158855246879451	9.84114475312055\\
-0.132175175231687	9.86782482476831\\
-0.105607156713064	9.89439284328694\\
-0.0791281053272401	9.92087189467276\\
-0.0527153892438813	9.94728461075612\\
-0.0263467123422732	9.97365328765773\\
0	10\\
0.0263467123422732	10.0263467123423\\
0.0527153892438814	10.0527153892439\\
0.0791281053272402	10.0791281053272\\
0.105607156713064	10.1056071567131\\
0.132175175231687	10.1321751752317\\
0.158855246879451	10.1588552468795\\
0.185671036089451	10.1856710360895\\
0.212646917525774	10.2126469175258\\
0.239808117294835	10.2398081172948\\
0.267180865704145	10.2671808657041\\
0.294792563997342	10.2947925639973\\
0.322671967867415	10.3226719678674\\
0.35084939101406	10.3508493910141\\
0.379356932587402	10.3793569325874\\
0.408228733076413	10.4082287330764\\
0.437501264092247	10.4375012640922\\
0.467213658611261	10.4672136586113\\
0.497408089641652	10.4974080896417\\
0.528130207043046	10.528130207043\\
0.559429644469353	10.5594296444694\\
0.591360611268586	10.5913606112686\\
0.623982587858245	10.6239825878582\\
0.657361147874823	10.6573611478748\\
0.691568936647646	10.6915689366476\\
0.726686843800421	10.7266868438004\\
0.762805418784888	10.7628054187849\\
0.800026592971978	10.800026592972\\
0.838465792122455	10.8384657921225\\
0.878254550923455	10.8782545509235\\
0.919543780240826	10.9195437802408\\
0.962507893023184	10.9625078930232\\
1.00735007450321	11.0073500745032\\
1.05430909929714	11.0543090992971\\
1.10366827295606	11.1036682729561\\
1.15576734289429	11.1557673428943\\
1.21101864209398	11.211018642094\\
1.26992940170539	11.2699294017054\\
1.33313328327202	11.333133283272\\
1.40143609274796	11.401436092748\\
1.47588404882448	11.4758840488245\\
1.55786934982539	11.5578693498254\\
1.64930035371341	11.6493003537134\\
1.75289015787549	11.7528901578755\\
1.87267775907536	11.8726777590754\\
2.01504837333302	12.015048373333\\
2.19095825718117	12.1909582571812\\
2.42158470939533	12.4215847093953\\
2.75650852190947	12.7565085219095\\
3.36492999890722	13.3649299989072\\
};
\addlegendentry{Q-Q Plot};

\end{axis}
\end{tikzpicture}%
   \caption{Q-Q Plot: \( (t_{5}+10) \) versus \( t_{5} \)}
   \label{fig:nct5qq}
\end{figure}
\begin{figure}[tbp]
   \centering
   % This file was created by matlab2tikz. 
%
%The latest updates can be retrieved from
%  http://www.mathworks.com/matlabcentral/fileexchange/22022-matlab2tikz-matlab2tikz
%where you can also make suggestions and rate matlab2tikz.
%
\definecolor{mycolor1}{rgb}{0.00000,0.44700,0.74100}%
%
\begin{tikzpicture}[scale=0.5]

\begin{axis}[%
width=6.028in,
height=4.754in,
at={(1.011in,0.642in)},
scale only axis,
xmin=-4,
xmax=4,
xlabel={\( t_{5} \) Quantiles},
ymin=-40,
ymax=40,
ylabel={\( 10 \times t_{5} \) Quantiles},
axis background/.style={fill=white},
axis x line*=bottom,
axis y line*=left,
legend style={at={(0.22,0.8)},anchor=south west,legend cell align=left,align=left,draw=white!15!black}
]
\addplot [color=red,solid]%dashdotted,forget plot
  table[row sep=crcr]{%
-3.36492999890722	-33.6492999890722\\
3.36492999890722	33.6492999890722\\
};
\addlegendentry{line passing through 1st and 3rd quartiles};

%% 
 % \addplot [color=red,solid,forget plot]
 %   table[row sep=crcr]{%
 % -0.717907367012227    -7.17907367012227\\
 % 0.717907367012227    7.17907367012227\\
 % };
 %%

\addplot [color=mycolor1,only marks,mark=+,mark options={solid,draw=blue}]%,forget plot
  table[row sep=crcr]{%
-3.36492999890722	-33.6492999890722\\
-2.75650852190947	-27.5650852190947\\
-2.42158470939533	-24.2158470939533\\
-2.19095825718118	-21.9095825718118\\
-2.01504837333302	-20.1504837333302\\
-1.87267775907536	-18.7267775907536\\
-1.75289015787549	-17.5289015787549\\
-1.64930035371341	-16.4930035371341\\
-1.55786934982539	-15.5786934982539\\
-1.47588404882448	-14.7588404882448\\
-1.40143609274796	-14.0143609274796\\
-1.33313328327202	-13.3313328327202\\
-1.26992940170539	-12.6992940170539\\
-1.21101864209398	-12.1101864209398\\
-1.15576734289429	-11.5576734289429\\
-1.10366827295606	-11.0366827295606\\
-1.05430909929714	-10.5430909929714\\
-1.00735007450321	-10.0735007450321\\
-0.962507893023184	-9.62507893023184\\
-0.919543780240827	-9.19543780240827\\
-0.878254550923454	-8.78254550923454\\
-0.838465792122455	-8.38465792122455\\
-0.800026592971978	-8.00026592971978\\
-0.762805418784888	-7.62805418784888\\
-0.726686843800421	-7.26686843800421\\
-0.691568936647646	-6.91568936647646\\
-0.657361147874823	-6.57361147874823\\
-0.623982587858245	-6.23982587858245\\
-0.591360611268586	-5.91360611268586\\
-0.559429644469353	-5.59429644469353\\
-0.528130207043046	-5.28130207043046\\
-0.497408089641652	-4.97408089641652\\
-0.467213658611261	-4.67213658611261\\
-0.437501264092247	-4.37501264092247\\
-0.408228733076414	-4.08228733076414\\
-0.379356932587402	-3.79356932587402\\
-0.35084939101406	-3.5084939101406\\
-0.322671967867415	-3.22671967867415\\
-0.294792563997342	-2.94792563997342\\
-0.267180865704145	-2.67180865704145\\
-0.239808117294835	-2.39808117294835\\
-0.212646917525774	-2.12646917525774\\
-0.185671036089451	-1.85671036089451\\
-0.158855246879451	-1.58855246879451\\
-0.132175175231687	-1.32175175231687\\
-0.105607156713064	-1.05607156713064\\
-0.0791281053272401	-0.791281053272401\\
-0.0527153892438813	-0.527153892438813\\
-0.0263467123422732	-0.263467123422732\\
0	0\\
0.0263467123422732	0.263467123422732\\
0.0527153892438814	0.527153892438814\\
0.0791281053272402	0.791281053272402\\
0.105607156713064	1.05607156713064\\
0.132175175231687	1.32175175231687\\
0.158855246879451	1.58855246879451\\
0.185671036089451	1.85671036089451\\
0.212646917525774	2.12646917525774\\
0.239808117294835	2.39808117294835\\
0.267180865704145	2.67180865704145\\
0.294792563997342	2.94792563997342\\
0.322671967867415	3.22671967867415\\
0.35084939101406	3.5084939101406\\
0.379356932587402	3.79356932587402\\
0.408228733076413	4.08228733076413\\
0.437501264092247	4.37501264092247\\
0.467213658611261	4.67213658611261\\
0.497408089641652	4.97408089641652\\
0.528130207043046	5.28130207043046\\
0.559429644469353	5.59429644469353\\
0.591360611268586	5.91360611268586\\
0.623982587858245	6.23982587858245\\
0.657361147874823	6.57361147874823\\
0.691568936647646	6.91568936647646\\
0.726686843800421	7.26686843800421\\
0.762805418784888	7.62805418784888\\
0.800026592971978	8.00026592971978\\
0.838465792122455	8.38465792122455\\
0.878254550923455	8.78254550923455\\
0.919543780240826	9.19543780240826\\
0.962507893023184	9.62507893023184\\
1.00735007450321	10.0735007450321\\
1.05430909929714	10.5430909929714\\
1.10366827295606	11.0366827295606\\
1.15576734289429	11.5576734289429\\
1.21101864209398	12.1101864209398\\
1.26992940170539	12.6992940170539\\
1.33313328327202	13.3313328327202\\
1.40143609274796	14.0143609274796\\
1.47588404882448	14.7588404882448\\
1.55786934982539	15.5786934982539\\
1.64930035371341	16.4930035371341\\
1.75289015787549	17.5289015787549\\
1.87267775907536	18.7267775907536\\
2.01504837333302	20.1504837333302\\
2.19095825718117	21.9095825718117\\
2.42158470939533	24.2158470939533\\
2.75650852190947	27.5650852190947\\
3.36492999890722	33.6492999890722\\
};
\addlegendentry{Q-Q Plot};

\end{axis}
\end{tikzpicture}%
   \caption{Q-Q Plot: \( 10*t_{5} \) versus \( t_{5} \)}
   \label{fig:scaledt5qq}
\end{figure}

Moving forward, we note that all of the distributions considered have unbounded support, so it is
just silly to attempt to cover quantiles over the entire support of the distributions.  In these
figures I have plotted points at 1\% increments starting at the 1\% quantile through to the 99\
quantile, for each distribution.  Moreover, each figure contains a solid red line which passes
through the points defined by the first and third quartiles of each distribution.  For ease of
reference we shall, hereafter, simply refer to this as the red line.\footnote{\matlab actually
provides a solid line joining these two points and then extrapolates that line to the extremes of
the plot (which is done using a dash-dot pattern).  In my plots I have deleted the shorter line
segment (as it isn't actually visible under the Q-Q plot) and converted the dash-dot pattern to a
solid line.} The red line serves no real purpose beyond acting as a visual aid to look at the
`straightness' of the Q-Q plot.  The plots of quantile pairs are then marked by blue `+' symbols.
All of the relevant commands are contained in the file \textbf{Exercise7.m}.  Note that I edited the
various .tex files created by the \texttt{matlab2tikz} command to change axis labels and to modify
the legend.  There were some other changes made to get things working.  I would encourage you to run
\textbf{Exercise7.m} and compare the output that you obtain with that used in generating this
document, to see the sorts of changes that I have made.  All of the \LaTeX\ source files are
available as a .zip file on the LMS.

Looking now at the patterns in the plots it makes sense to start at Figure \ref{fig:t5qq}, which
plots the quantiles of a \( t_{5} \) distribution against themselves.  As you would expect, the `+'
symbols all lie on the red line which, in this case, is a 45 degree line.  In practice if you were
to generate a sample of random draws from some distribution then you almost certainly would not
observe such a perfect correspondence.  In particular, the extremes of the distribution are often
under-sampled and so you may get departures there.

Excluding the bonus plots for the moment, the remaining distributions can be broken up into two
groups.  The first group consists of \( t \) distributions with larger degrees of freedom that the
reference distribution, namely the standard Normal (\( t_{\infty} \)), the \( t_{20} \) and the \(
t_{10} \).  These distributions all have relatively thin tails and so their quantiles are somewhat
more tightly clustered about zero than those of the \( t_{5} \) distribution.  Looking at Figures
\ref{fig:normqq}--\ref{fig:t10qq}, we see essentially the same pattern in each.  To the left of
zero, for a given percentage, the quantile for a \( t_{5} \) distribution is a smaller (more
negative) number than that for the thinner-tailed distributions, e.g.\ the 1\% quantiles for the \(
t_{5} \) and the \( \N(0,1) \) are approximately \( -3.3649 \) and \( -2.3263 \), respectively.
Similarly, to the right of zero, the quantiles of the \( t_{5} \) distribution are larger numbers
than those of the thinner-tailed distributions, e.g.\ the 99\% quantiles for the \( t_{5} \) and the
\( \N(0,1) \) are approximately 3.3649 and 2.3263, respectively.  Because all of these distributions
are symmetric about zero we see that they each have their 50\% quantile at zero.

The other group of symmetric distributions is comprised of \( t \) distributions with fewer degrees
of freedom than 5, namely the \( t_{2} \) and the \( t_{1} \) or Cauchy distributions, which have
fatter tails than does the \( t_{5} \) distribution.  Here we might expect to see the patterns
reversed in the tails of the distributions but, again, similarity around the point of symmetry.  Not
surprisingly this is exactly what we see.  Interestingly, the evidence suggests that a Cauchy
distribution has much fatter tails relative to a \( t_{5} \) distribution than does a \( t_{5} \)
relative to a standard Normal distribution.

Figures \ref{fig:chisq3qq} and \ref{fig:chisq30qq} plot the quantiles of skewed distributions
against the quantiles of the \( t_{5} \) distribution, which is symmetric about zero.  Observe that
the skewed distributions have been centred so that these distributions also have zero mean.  This is
done by subtracting their respective degrees of freedom which, as we know, is the mean of a
chi-squared random variable.  For both of the skewed distributions there is a much greater
probability mass down towards the lower bound of their support than for the \( t_{5} \) distribution
and this is far more pronounced at smaller degrees of freedom.  Indeed, we know that as \(
\nu\to\infty \), \( (\chi^{2}_{\nu}-\nu)/\sqrt{2\nu}\to\N(0,1) \) and so, at larger degrees of
freedom the support of the skewed distribution is increasing and the quantity \( \chi^{2}_{\nu}-\nu
\) is becoming increasingly symmetric about zero.  This is reflected in the Q-Q plot in Figure
\ref{fig:chisq30qq}.  As we move into the right-hand tails of these distributions, all the
probability mass at the lower end of the support means that the right-hand tails of the distribution
must become thin.  Just how thin relative to the \( t_{5} \) distribution differs with the degrees
of freedom.  In particular, we see that for the larger the degrees of freedom of the chi-squared
variate, the smaller the probability mass at the lower bound of the support and hence the thicker
the right-hand tail of the distribution.  (You really need to get this off the scale of the \( y
\)-axis as the location of the red line vis-a-vis the Q-Q plot is not helpful in this regard.)  For
example, the 99\% quantile for \( \chi^{2}_{30}-30 \) distribution is approximately 20.8922, that of
the \( \chi^{2}_{3}-3 \) distribution is approximately 8.3449, whereas that for the \( t_{5} \)
distribution is only 3.3649 (approximately).  We conclude that the probability mass around zero for
the \( t_{5} \) distribution is larger than that at the lower bound of the support of the skewed
distributions.

Our final experiments involve transformations of the \( t_{5} \) variates by either the addition of
a positive constant (10 in Figure \ref{fig:nct5qq}) or by scaling the variate by a positive
constant (10 in Figure \ref{fig:scaledt5qq}).  The remarkable thing is that, as was seen in Figure
\ref{fig:t5qq}, the Q-Q plots continue to live on a straight-line, albeit no longer a 45 degree
line.  Nevertheless, affine transformations of random variables, transformations of the type \(
Y=\alpha+\beta X \) will have the same impact on the quantiles of the distribution.  More
complicated transformations will have different impacts.  For example, the square of a \( t_{5} \)
random variable won't be chi-squared, but it will be something similar and so Figures
\ref{fig:chisq3qq} and \ref{fig:chisq30qq} provide some insight to what might happen.


\section*{Exercise}
\begin{enumerate}
   \item  
   \begin{enumerate}
      \item Consider the model \( y=x\beta+u \) where the observed variables \( y \) and \( x \), 
            and the unobserved disturbance \( u \) are all \( n \)-vectors.  Show that the 
            OLS estimator for \( \beta \) is 
            \begin{equation}
               \hat{\beta}=\frac{x'y}{x'x}.
               \label{eq:m1}
            \end{equation}
            \solution
            The OLS estimator is defined according to
            \[ 
               \hat{\beta}=\argmin_{\beta}(y-x\beta)'(y-x\beta)
               =\argmin_{\beta}[y'y-2x'y\beta+x'x\beta^{2}]=\argmin_{\beta}S(\beta), 
            \]
            say.  Now, \( S(\beta) \) is everywhere continuous and differentiable and so calculus is our 
            friend here, meaning that our solution will satisfy the first order condition
            \[
               0=\left.\frac{\mathrm{d} S(\beta)}{\mathrm{d} \beta}\right|_{\beta=\hat{\beta}} 
                =-2x'y+2x'x\hat{\beta}\implies x'x\hat{\beta}=x'y\implies\hat{\beta}=\frac{x'y}{x'x},
            \]
            as required.  
            
      \item Suppose that, in \eqref{eq:m1}, \( y\sim\N(0,I_{n}) \) and that \( y \) and \( x \) are 
            statistically independent.  Then, show that 
            \[ \alpha=\frac{x'y}{(x'x)^{1/2}}\sim\N(0,1) \]
            and that \( \alpha \) is independent of \( x \).
            
            \solution If \( y \) and \( x \) are statistically independent then the conditional
            distribution of \( y \) given \( x \) is the same as the unconditional distribution of
            \( y \).  That is, 
            \[ 
               y\sim\N(0,I_{n})\implies y\mid x\sim\N(0,I_{n}).
            \]
            Consequently,
            \[ 
               \alpha\mid x=\left.\frac{x'y}{(x'x)^{1/2}}\right| x\sim
               \N\left(\frac{x'0}{(x'x)^{1/2}},\frac{x'}{(x'x)^{1/2}}I_{n}\left(\frac{x'}{(x'x)^{1/2}}\right)'\right)
               =\N(0,1).
            \]
            Noting that the conditional distribution of \( \alpha\mid x \) does not depend on \( x 
            \), it follows that it is also the unconditional distribution of \( \alpha \) and that 
            \( \alpha \) is independent of \( x \).
            
      \item Further suppose that \( x\sim\N(0,I_{n}) \).  Since \( \hat{\beta}=\alpha/(x'x)^{1/2} 
            \), where \( \alpha\sim\N(0,1) \) and is independent of \( x \), and since \( 
            x'x\sim\chi^{2}_{n} \), deduce that \( n^{1/2}\hat{\beta}\sim t_{n} \).
            
            \solution
            We have seen that \( \alpha \) is independent of \( x \), because its conditional density is not a 
            function of \( x \).  Consequently, it will be independent of functions of \( x \) and, 
            specifically, of \( (x'x)^{1/2} \).  Given that \( x\sim\N(0,I_{n}) \) we know that \( 
            x'x\sim\chi^{2}_{n} \).  Therefore,
            \[ 
               n^{1/2}\hat{\beta}=\frac{n^{1/2}\alpha}{(x'x)^{1/2}}=\frac{\alpha}{(x'x/n)^{1/2}}
               \sim\frac{N(0,1)}{\sqrt{\chi^{2}_{n}/n\,}}=t_{n},
            \]
            as required, where the final equality follows as a consequence of the numerator and 
            denominator.
   \end{enumerate}

   \item Suppose that \( X=[X_{1},X_{2},X_{3}]'\sim\N(\mu,\Sigma) \).  If 
         \[ 
            \Sigma=
            \begin{bmatrix}
               1 & \rho & \rho^{2} \\
               \rho & 1 & 0 \\
               \rho^{2} & 0 & 1
            \end{bmatrix}
         \]
         show that the conditional distribution of \( [X_{1},X_{2}]' \) given \( X_{3} \) has mean 
         vector 
         \begin{gather*}
            [\mu_{1}+\rho^{2}(x_{3}-\mu_{3}),\mu_{2}]' \\
            \intertext{and covariance matrix}
            \begin{bmatrix}
               1-\rho^{4} & \rho \\
               \rho & 1
            \end{bmatrix}.
         \end{gather*}
         \solution 
         Let \( Y_{1}=[X_{1},X_{2}]' \) and \( Y_{2}=X_{3} \), so that \( Y=[Y_{1}',Y_{2}]'=X \).
         Equally, let
         \[ 
            \Omega=
            \begin{bmatrix}
               \Omega_{11} & \omega_{12} \\
               \omega_{21} & \omega_{22}
            \end{bmatrix},
         \]
         where \( \omega_{22}=1 \), \( \omega_{12}=[\rho^{2},0]'=\omega_{21}' \) and
         \[ 
            \Omega_{11}=\begin{bmatrix} 1 & \rho \\ \rho & 1 \end{bmatrix},
         \] 
         so that \( \Omega=\Sigma \).  If we partition \( \E{X}=\mu=[\mu_{1},\mu_{2},\mu_{3}]' \),
         say, and \( \E{Y}=\mu=[\mu_{Y_{1}}',\mu_{Y_{2}}]' \), with \( \mu_{Y_{1}}=[\mu_{1},\mu_{2}]'
         \) and \( \mu_{Y_{2}}=\mu_{Y_{2}} \), we see that this is less a transformation than a
         re-grouping of the elements of \( X \).  (Alternatively, it is a transformation for which
         the Jacobian of transformation is unity.)  From our results on the conditional distribution
         of jointly normally distributed random variables, we see that \( Y_{1}\mid
         Y_{2}\sim\N\left(\mu_{Y_{1}\mid Y_{2}}, \Omega_{11\cdot 2}\right) \), where
         \begin{align*}
            \mu_{Y_{1}\mid Y_{2}} & =\mu_{Y_{1}}+\omega_{12}\omega_{22}^{-1}(Y_{2}-\mu_{Y_{2}})
            =
            \begin{bmatrix}
               \mu_{1} \\
               \mu_{2}
            \end{bmatrix}
            +
            \begin{bmatrix}
               \rho^{2} \\
               0
            \end{bmatrix}
            (1)^{-1}(X_{3}-\mu_{X_{3}})
            \\& 
            =
            \begin{bmatrix}
               \mu_{1}+\rho^{2}(X_{3}-\mu_{X_{3}}) \\
               \mu_{2}
            \end{bmatrix}
            \\
            \shortintertext{and}
            \Omega_{11\cdot 2} & =\Omega_{11}-\omega_{12}\omega_{22}^{-1}\omega_{21}
            =
            \begin{bmatrix}
               1 & \rho \\
               \rho & 1
            \end{bmatrix}
            -
            \begin{bmatrix}
               \rho^{2} \\
               0
            \end{bmatrix}
            (1)^{-1}[\rho^{2},0]
            =
            \begin{bmatrix}
               1-\rho^{4} & \rho \\
               \rho & 1
            \end{bmatrix},
         \end{align*}
         as required.

   \item If \( X_{1} \), \( X_{2} \), and \( X_{3} \) are iid.\ \( \N(\mu,\Sigma) \) \( p 
         \)-vectors, and if \( Y_{1}=X_{1}+X_{2} \), \( Y_{2}=X_{2}+X_{3} \), and \( 
         Y_{3}=X_{1}+X_{3} \), then obtain
         \begin{enumerate}
            \item the conditional distribution of \( Y_{1} \) given \( Y_{2} \); and
         
            \item the conditional distribution of \( Y_{1} \) given \( Y_{2} \) and \( Y_{3} \).
         \end{enumerate}
         
         Hint: In lectures you were given a formula for \( K^{-1} \), the inverse of a partitioned 
         matrix \( K \), where 
         \[
            K=
            \begin{bmatrix}
               A & B \\
               C & D
            \end{bmatrix}.
         \]
         This formula that was given is very general and works even when \( B \) and \( C \) are not square.  All
         it required was that \( A \) was non-singular.  However, when \( B \) and \( C \) are also
         non-singular there is considerable simplification available.  In our case there is even
         greater simplification arising from the facts that (i) \( A=D \) and \( B=C \), and (ii)
         that \( A=\alpha B \) for some scalar constant \( \alpha \).  Hence, we need the inverse of
         a matrix of the form
         \[ 
            K=
            \begin{bmatrix}
               \alpha B & B \\
               B & \alpha B
            \end{bmatrix},
         \]
         with \( B \) non-singular.  Now, it must be the case that \( KK^{-1}=I \).  If we partition \(
         K^{-1} \) conformably with \( K \) then we can write
         \[ 
            K^{-1}=
            \begin{bmatrix}
               E & F \\
               G & H
            \end{bmatrix},
         \]
         say, and so \( KK^{-1}=I \) implies a set of 4 equations in 4 unknowns \( (E,F,G,H) \):
         \begin{align*}
            \alpha BE+BG & =I, & \alpha BF+BH & =0, \\
            BE+\alpha BG & =0, & BF+\alpha BH & =I,
         \end{align*}
         From the zero equations we see that 
         \begin{align*}
            H & =-\alpha F, & E & =-\alpha G ,
         \end{align*}
         which we can substitute back into the identity equations to obtain
         \begin{gather*}
            -\alpha^{2}BG+BG=I \implies (1-\alpha^{2})BG=I \implies G=(1-\alpha^{2})^{-1}B^{-1},
            \\
            BF-\alpha^{2}BF=I \implies (1-\alpha^{2})BF=I \implies F=(1-\alpha^{2})^{-1}B^{-1}.
         \end{gather*}
         We can substitute these expressions for \( G \) and \( H \) into those for \( E \) and \( 
         F \) above to obtain
         \[
            K^{-1}=
            \begin{bmatrix}
               -\tfrac{\alpha}{1-\alpha^{2}}B^{-1} & \hphantom{-}\tfrac{1}{1-\alpha^{2}}B^{-1} \\[5pt]
               \hphantom{-}\tfrac{1}{1-\alpha^{2}}B^{-1} & -\tfrac{\alpha}{1-\alpha^{2}}B^{-1}
            \end{bmatrix}
         \]
         Note that \( K^{-1} \) retains the symmetry of \( K \) and, like \( K \), all four
         partitions of \( K^{-1} \) are also non-singular.  
         
         \solution
         Because \( X_{1} \), \( X_{2} \), and \( X_{3} \) are independent we know that \( 
         X=[X_{1}',X_{2}',X_{3}']'\sim\N\left(\delta,\Omega\right) \), where \( 
         \delta=[\mu',\mu',\mu']' \) and \( \Omega=\diag(\Sigma,\Sigma,\Sigma) \). Now, \( 
         Y=[Y_{1}',Y_{2}',Y_{3}']'=AX \), where
         \[ A=
            \begin{bmatrix}
               I_{p} & I_{p} & 0 \\
               0 & I_{p} & I_{p} \\
               I_{p} & 0 & I_{p}
            \end{bmatrix}.
         \]
         From the properties of Normal random variables we see that 
         \begin{align*}
            \E{Y} & =A\E{X}=A\delta=2\delta \\
            \shortintertext{and}
            \V{Y} & =A\V{X}A'=
            \begin{bmatrix}
               I_{p} & I_{p} & 0 \\
               0 & I_{p} & I_{p} \\
               I_{p} & 0 & I_{p}
            \end{bmatrix}
            \begin{bmatrix}
               \Sigma & 0 & 0 \\
               0 & \Sigma & 0 \\
               0 & 0 & \Sigma
            \end{bmatrix}
            \begin{bmatrix}
               I_{p} & 0 & I_{p} \\
               I_{p} & I_{p} & 0 \\
               0 & I_{p} & I_{p}
            \end{bmatrix}
            \\
            & =
            \begin{bmatrix}
               2\Sigma & \Sigma & \Sigma \\
               \Sigma & 2\Sigma & \Sigma \\
               \Sigma & \Sigma & 2\Sigma
            \end{bmatrix}
            =2I_{3}\otimes\Sigma=\Omega\text{ (say)}.
         \end{align*}
         
         \begin{enumerate}
            \item We can obtain the conditional distribution of \( Y_{1} \) given \( Y_{2} \) from 
                  the joint distribution of \( Y_{1} \) and \( Y_{2} \), which is just the marginal 
                  distribution of \( Y_{1} \) and \( Y_{2} \) from the joint distribution of \( Y 
                  \).  That is, 
                  \begin{gather*}
                     \begin{bmatrix}
                        Y_{1} \\
                        Y_{2}
                     \end{bmatrix}\sim\N\left(
                     2
                     \begin{bmatrix}
                        \mu \\
                        \mu
                     \end{bmatrix},
                     \begin{bmatrix}
                        2\Sigma & \Sigma \\
                        \Sigma & 2\Sigma
                     \end{bmatrix}\right).
                     \\
                     \shortintertext{Hence, using the formulae provided in the Solution to Question 
                     2, with appropriate partitioning of the parameters here,}
                     Y_{1}\mid Y_{2}\sim\N\left(2\mu+\Sigma(2\Sigma)^{-1}(y_{2}-2\mu),
                     2\Sigma-\Sigma(2\Sigma)^{-1}\Sigma\right)=
                     \N\left(\tfrac{1}{2}(y_{2}+2\mu),\tfrac{3}{2}\Sigma\right).
                  \end{gather*}
         
            \item We can read the condition distribution of \( Y_{1} \) given \( Y_{2} \) and \( 
                  Y_{3} \) directly from the joint distribution of \( Y \) as follows: \( Y_{1}\mid 
                  Y_{2},Y_{3}\sim\N\left(\mu_{1\cdot 23},\Omega_{11\cdot(23)}\right) \), say, where
                  \begin{align*}
                     \mu_{1\cdot 23} & =2\mu+[\Sigma,\Sigma]
                     \begin{bmatrix}
                        2\Sigma & \Sigma \\
                        \Sigma & 2\Sigma
                     \end{bmatrix}^{-1}\left(
                     \begin{bmatrix}
                        y_{2} \\ y_{3}
                     \end{bmatrix}
                     -2
                     \begin{bmatrix}
                        \mu \\ \mu
                     \end{bmatrix}\right)
                     =\tfrac{1}{3}(y_{2}+y_{3}+2\mu)
                     \\
                     \shortintertext{and}
                     \Omega_{11\cdot(23)} & =2\Sigma-[\Sigma,\Sigma]
                     \begin{bmatrix}
                        2\Sigma & \Sigma \\
                        \Sigma & 2\Sigma
                     \end{bmatrix}^{-1}
                     \begin{bmatrix}
                        \Sigma \\
                        \Sigma
                     \end{bmatrix}=\tfrac{4}{3}\Sigma.
                  \end{align*}
                  Note that this conditional variance is smaller than the unconditional variance of 
                  \( Y_{1} \) in the sense that \( \V{Y_{1}}-\V{Y_{1}\mid 
                  Y_{2},Y_{3}}=2\Sigma-\tfrac{4}{3}\Sigma=\tfrac{2}{3}\Sigma>0 \).
         \end{enumerate}
\end{enumerate}
Note that both of the conditional variances obtained in Question 3 are smaller than the unconditional variance of 
\( Y_{1} \):
\begin{enumerate}
   \item \( \V{Y_{1}}-\V{Y_{1}\mid Y_{2}}=2\Sigma-\tfrac{3}{2}\Sigma=\tfrac{1}{2}\Sigma>0 \). 

   \item \( \V{Y_{1}}-\V{Y_{1}\mid Y_{2},Y_{3}}=2\Sigma-\tfrac{4}{3}\Sigma=\tfrac{2}{3}\Sigma>0 \).
\end{enumerate}
Further note that the more you condition on the greater the variance reduction, that is 
\[ 
   \V{Y_{1}\mid Y_{2}}-\V{Y_{1}\mid Y_{2},Y_{3}}=\tfrac{3}{2}\Sigma-\tfrac{4}{3}\Sigma=\tfrac{1}{6}\Sigma>0 .
\]

%% 
 % \vspace{2in}
 % \begin{center}
 %    Your answers to the Exercise should be submitted to the LMS no later than 3:00pm, Friday 20
 %    March.  They should be typed.
 %     
 %    No late exercises are accepted but an incomplete exercise is better than nothing.
 % \end{center}
 %%

\end{document}
