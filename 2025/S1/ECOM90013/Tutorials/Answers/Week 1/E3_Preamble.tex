%\DocumentMetadata{testphase=phase-III}
\documentclass[12pt,a4paper]{article}
\usepackage{amsmath,amssymb,amsthm,booktabs,cancel,color,enumitem,graphicx,listings,mathtools,metalogo,%tikz,mathrsfs,
multicol,pdfpages,pgfplots,rotating,subfig,widebar,xparse,xspace}

\pgfplotsset{compat=1.15}
\usetikzlibrary{intersections}

%\usepackage{eucal}
\usepackage[scr]{rsfso}
\usepackage[margin=1in]{geometry}
\usepackage[comma,longnamesfirst,round]{natbib}
%
\usepackage[colorlinks=true,% If false then links black, but set in ugly red boxes
                            % If true then the following colour scheme is used.
            citecolor=blue,% The named colours include: black, white, red, yellow
            linkcolor=red,%                             blue, green, cyan, magenta
            anchorcolor=yellow,%
            filecolor=cyan,% Can set all all colors to a single color using the following command
            menucolor=red,% allcolors=<colour>.  For example, allcolors=black would be idea for
            runcolor=filecolor,% printing hard copies.  Simply uncomment the command on the following
            urlcolor=magenta,% line if you want this.
            %allcolors=black,
            pagebackref=true% In the bibliography gives citation page references. To turn on set
                            % to true
            ]{hyperref}

%% 
 % \usepackage[tagged, highstructure]{accessibility}
 % \usepackage{axessibility}
 %%
            
%
% Theorems and Remarks
%
\theoremstyle{definition}
\newtheorem{ass}{Assumption}

\theoremstyle{plain}
\newtheorem*{Thm}{Theorem}
\newtheorem{thm}{Theorem}
\theoremstyle{remark}
\newtheorem*{remark}{Remark}
\newtheorem{rem}{Remark}
\newtheorem{corr}{Corollary}
\theoremstyle{definition}
%\newtheorem{Def}{Definition}
\newtheorem*{deff}{Definition}

\newtheoremstyle{example}% name
  {9pt}%      Space above, empty = `usual value'
  {9pt}%      Space below
  {\normalfont} %\renewcommand{\myfiletype}{loe}}% Body font
  {}%         Indent amount (empty = no indent, \parindent = para indent)
  {\bfseries}% Thm head font
  {.}%        Punctuation after thm head
  {\newline}% Space after thm head: \newline = linebreak
  {\thmname{#1}\thmnumber{ #2.}\thmnote{ #3}}%Thm head spec
  
\theoremstyle{example}
\newtheorem{example}{Example}

\newtheoremstyle{Def}% name
  {9pt}%      Space above, empty = `usual value'
  {9pt}%      Space below
  {\normalfont}%\addcontentsline{lod}{Def}{\protect\numberline{\theDef}y}} %\renewcommand{\myfiletype}{lod}}% Body font
  {}%         Indent amount (empty = no indent, \parindent = para indent)
  {\bfseries}% Thm head font
  {.}%        Punctuation after thm head
  {\newline}% Space after thm head: \newline = linebreak
  {\thmname{#1}\thmnumber{ #2.}\thmnote{ #3}}%         Thm head spec
  
\theoremstyle{Def}
\newtheorem{Def}{Definition}

\newcommand{\abs}[1]{\ensuremath{\operatorname{abs}\left(#1\right)}}
\newcommand{\ans}{\newline\textit{\textcolor{red}{Answer:}}\newline}
\newcommand{\ant}{\textit{\textcolor{red}{Answer:}}\newline}
\makeatletter
   \def\argmax{\mathop{\operator@font argmax}}
   \def\argmin{\mathop{\operator@font argmin}}
\makeatother
%\newcommand{\argmax}{\arg\max}
%\newcommand{\argmin}{\arg\min}
\newcommand{\Beta}[2]{\ensuremath{\operatorname{B}\left(#1,#2\right)}}
\def\BibTeX{{\rm B\kern-.05em{\sc i\kern-.025em b}\kern-.08em
    T\kern-.1667em\lower.7ex\hbox{E}\kern-.125emX}}
\newcommand{\cmd}[1]{\mbox{\texttt{\symbol{'134}#1}}}
\newcommand{\Co}[1]{\ensuremath \mathrm{Cov}_{0}\left[#1\right]}
\newcommand{\cor}[2]{\ensuremath{\operatorname{cor}\left(#1,#2\right)}}
\newcommand{\cov}[2]{\ensuremath{\operatorname{Cov}\left[#1,#2\right]}}
\DeclareMathOperator{\diag}{diag}
\newcommand{\dr}{\ensuremath{\mrmd r}}
\newcommand{\ds}{\ensuremath{\mrmd s}}
\newcommand{\dt}{\ensuremath{\mrmd t}}
\newcommand{\dw}{\ensuremath{\mrmd w}}
\newcommand{\dx}{\ensuremath{\mrmd x}}
\newcommand{\dy}{\ensuremath{\mrmd y}}
\newcommand{\dz}{\ensuremath{\mrmd z}}
\DeclareMathOperator{\etr}{etr}
\newcommand{\E}[1]{\ensuremath{\operatorname{E}\left[#1\right]}}
\newcommand{\Eo}[1]{\ensuremath \mathrm{E}_{0}\left[#1\right]}
\newcommand{\Eq}[1]{\begin{align*}#1\end{align*}}
\newcommand{\Ex}[2]{\ensuremath{\operatorname{E}_{#1}\left[#2\right]}}
\newcommand{\Gammafrac}[2][2]{\ensuremath{\Gamma\left(\tfrac{#2}{#1}\right)}}%\Gamma[x]{y} yields Gamma(y/x)
\newcommand{\gfn}[1]{\ensuremath{\Gamma\left(#1\right)}}
\newcommand{\half}[1]{\ensuremath{\frac{#1}{2}}}
\newcommand{\hyperg}[2]{\ensuremath{\prescript{}{#1}{F}^{}_{#2}}}
\newcommand{\im}[1]{\ensuremath{\operatorname{Im}\left(#1\right)}}
\newcommand{\intinfty}{\int^{\infty}_{0}}
\newcommand{\matlab}{\textsc{Matlab}\xspace}
\newcommand{\mg}{\textit{\textcolor{red}{Marking Guide}}:\xspace\newline}
\newcommand{\mgt}{\newline\textit{\textcolor{blue}{Marking Guide:}}\newline}
\newcommand{\Mo}[1]{\ensuremath \mathrm{MSE}_{0}\left[#1\right]}
\newcommand{\mrmd}{\ensuremath{\,\mathrm{d}}}
\newcommand{\myrego}{\,\raisebox{1.5pt}{\scalebox{0.81}{\textregistered}}\xspace}
\DeclareMathOperator{\N}{N}
\newcommand{\ninf}[1][n]{#1\to\infty}
\newcommand{\param}[2][n]{\ensuremath{#2_{1} #2_{2}\ldots #2_{#1}}}
\newcommand{\params}[2][n]{\ensuremath{#2_{1},#2_{2},\ldots,#2_{#1}}}
\newcommand{\paramsj}[3][j]{\ensuremath{\left(#2_{1}\right)_{#1}\left(#2_{2}\right)_{#1}\ldots\left(#2_{#3}\right)_{#1}}}
\newcommand{\pdflatex}{pdf\LaTeX\xspace}
\makeatletter
   \def\plim{\mathop{\operator@font plim}}
\makeatother
\newcommand{\Poch}[2]{\ensuremath{\left(#1\right)_{#2}}}
\newcommand{\Prob}[1]{\ensuremath{\operatorname{Pr}\left(#1\right)}}
\newcommand{\re}[1]{\ensuremath{\operatorname{Re}\left(#1\right)}}
\newcommand{\sgn}[1]{\ensuremath{\operatorname{sgn}\left(#1\right)}}
%Duplicated commands:  Settle on a choice.  No. They do different things!
\newcommand{\solution}{\textit{\textcolor{red}{Solution}}:\xspace\newline}
\newcommand{\sol}{\textcolor{red}{Solution:}\xspace}
%
\DeclareDocumentCommand{\sumj}{ O{j} O{0} }{\ensuremath{\sum_{#1=#2}^{\infty}\,}}
\newcommand{\tod}{\ensuremath{\overset{d}{\to}}}
\DeclareMathOperator{\tr}{tr}
\DeclareMathOperator{\myvec}{vec}
\DeclareMathOperator{\vech}{vech}
% More duplicates: Choice as been made?
\newcommand{\V}[1]{\ensuremath{\operatorname{Var}\left[#1\right]}}
\newcommand{\Vhat}[1]{\ensuremath{\operatorname{\protect\widehat{V}}\left[#1\right]}}
\newcommand{\Vo}[1]{\ensuremath \mathrm{V}_{0}\left[#1\right]}
%\newcommand{\var}[1]{\ensuremath{\operatorname{Var}\left(#1\right)}}
\newcommand{\Vx}[2]{\ensuremath{\operatorname{Var}_{#1}\left[#2\right]}}

%Boxes
\newsavebox{\mystrutbox}
\savebox{\mystrutbox}{\rule{0mm}{0.6cm}}
\newcommand{\mystrut}{\usebox{\mystrutbox}}

\newlength{\twosuper}
\settowidth{\twosuper}{\textsuperscript{2}}

%
% List Management
%
\makeatletter
   \renewcommand{\p@enumii}{\relax}
   \def\plim{\mathop{\operator@font plim}}
\makeatother
\renewcommand{\theenumii}{(\alph{enumii})}
\renewcommand{\labelenumii}{\theenumii}
\makeatletter
   \renewcommand{\p@enumii}{\theenumi}
   \renewcommand{\p@enumiii}{\theenumi\theenumii}
\makeatother

% 
% An extensions to amsmath matrix environments
% (adds the same column specification options that array has)
% originally from http://texblog.net/latex-archive/maths/amsmath-matrix/
% 
\makeatletter
\renewcommand*\env@matrix[1][*\c@MaxMatrixCols c]{%
  \hskip -\arraycolsep
  \let\@ifnextchar\new@ifnextchar
  \array{#1}}
\makeatother


\thispagestyle{empty}
\bibliographystyle{Skeels}

\title{ECOM40006/ECOM90013 Econometrics 3 \\
        Department of Economics \\
        University of Melbourne}
\date{Semester 1, \the\year}

