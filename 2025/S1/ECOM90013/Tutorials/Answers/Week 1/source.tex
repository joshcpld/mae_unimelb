%% 
 % Assigment Template
 % 
 % Usage: You need to supply some details about the document in the
 % preamble (before the \begin{document} command) and then the
 % assignment content in the body of the document (before the
 % \begin{document} command).
 % 
 %%
\documentclass[12pt,a4paper]{article}
\usepackage{amsmath,amssymb,amsthm} \allowdisplaybreaks
\usepackage{fancyvrb,graphicx,listings,url} \VerbatimFootnotes
\usepackage[hmargin=1.2in,vmargin=1in]{geometry}
\usepackage[comma,longnamesfirst]{natbib} \bibliographystyle{plainnat}
%
% Add requested information between the parentheses { } below
\newcommand\yourName{Chris Skeels} % If nothing else you know your own name.
\newcommand\studentID{You know who I am} 
% If all else fails look at your student ID card. 
\newcommand\assignmentName{Week 1 Exercise} 
% Name of assignment, e.g. Assignment 1, or essay title.
\newcommand\submissionDate{\today} 
% The default date is the one on which the file is last compiled.  You
% can change this to, say, the date the file was last modified if
% desired.
\newcommand\subjectCode{ECOM40006/ECOM90013} 
% This will be something along the lines of ECOMxxxxx or ECONxxxxx.
\newcommand\subjectName{Econometrics 3} 
% The subject name without the subject code, e.g. Econometric Techniques.
\newcommand\yourBibliography{CHRIS} 
% Supply name of .bib file, but without the .bib extension
\newcommand{\tl}[1]{\Large\textbf{#1}}
%
% End of Preamble
% 
\begin{document} 
%
% Create Coversheet
% 
\thispagestyle{empty} \null \vskip 2em \begin{tabbing}
\null\hspace*{3.6in}\=\kill \tl{\subjectCode} \> \tl{\subjectName}
\\[\baselineskip] \tl{\studentID} \> \tl{\yourName} \\[\baselineskip]
\tl{\assignmentName} \end{tabbing}
\vspace{1cm} \noindent\textbf{PLAGIARISM}\\[\baselineskip] Plagiarism
is the presentation by a student of an assignment which has in fact
been copied in whole or in part from another student's work, or from
any other source (E.g. published books or periodicals), without due
acknowledgement in the text.\bigskip

\noindent\textbf{COLLUSION}\\[\baselineskip] Collusion is the
presentation by a student of an assignment as his or her own which is
in fact the result in whole or part of unauthorised collaboration with
another person or persons.\bigskip

\noindent\textbf{DECLARATION}\\[\baselineskip] This assignment/essay
is the sole work of the author whose name appears on the Title Page
and contains no material which the author has previously submitted for
assessment at The University of Melbourne or elsewhere.  Also, to the
best of the author's knowledge and belief, the assignment/essay
contains no material previously published or written by another person
except where due reference is made in the text of the
assignment/essay.  I declare that I have read, and in undertaking this
research, have complied with the University's Code of Conduct for
Research.  I also declare that I understand what is meant by
plagiarism and that this is unacceptable; except where I have
expressly indicated otherwise.  This assignment/essay is my own work
and does not contain any plagiarised material in the form of
unacknowledged quotations or mathematical workings or in any other
form.  \vspace{0.5in}

\noindent I declare that this assignment/essay is my own work and does
not involve plagiarism or collusion.  \vspace{0.5in}

\noindent Signed:\hspace{1cm} \begin{minipage}[t]{0.2\textwidth}
	\begin{center}
		\mbox{X} \\
		\yourName \end{center}
\end{minipage}
\hfill Date: \submissionDate
%
% End Coversheet
\newpage
%
% Enter your text here
% 
The \LaTeX\ that I used to generate the Tutorial Exercise appears on
the next couple of pages.  My Bib\TeX\ entries appear immediately
below.\footnote{Note the use of a double hyphen to specify a page
range.  Strictly \verb+--+ prduces what is called an En-dash, which is
used for number ranges.  You use a single hyphen for a hyphenated
word, like non-singular, and a triple hyphen \verb+---+ to get an
Em-dash ---.  See
\url{https://www.thepunctuationguide.com/en-dash.html} for more on
these versatile creatures.  In \LaTeX\ there is also the minus sign,
which is what you get when you use a hyphen in math mode, e.g.,\ \(
x-y \).  These symbols are all different and they all have different
applications.}

\begin{lstlisting}
   @article{abadir99, 
      author = {Karim M. Abadir}, 
      title = {An Introduction to Hypergeometric Functions 
               For Economists}, 
      journal = {Econometric Reviews}, 
      year = 1999, 
      volume = 18, 
      number = 3, 
      pages = {287--330} 
   }

   @book{rao, author = {Calyampudi Radhakrishna Rao}, 
      title = {Linear Statistical Inference and Its Applications}, 
      edition = {Second},
      publisher = {John Wiley \& Sons, Inc.}, 
      year = 1973, 
      address = {New York} 
   } 
\end{lstlisting}

Finally, in order to do it all you need the file
\texttt{E3\_Preamble.tex} in which I store all sorts of information
that gets used in almost everything I produce for you.  I attach that
file along with this one in the solutions.  The end result is what
started all this.

\newpage

\begin{lstlisting}
%\DocumentMetadata{testphase=phase-III}
\documentclass[12pt,a4paper]{article}
\usepackage{amsmath,amssymb,amsthm,booktabs,cancel,color,enumitem,graphicx,listings,mathtools,metalogo,%tikz,mathrsfs,
multicol,pdfpages,pgfplots,rotating,subfig,widebar,xparse,xspace}

\pgfplotsset{compat=1.15}
\usetikzlibrary{intersections}

%\usepackage{eucal}
\usepackage[scr]{rsfso}
\usepackage[margin=1in]{geometry}
\usepackage[comma,longnamesfirst,round]{natbib}
%
\usepackage[colorlinks=true,% If false then links black, but set in ugly red boxes
                            % If true then the following colour scheme is used.
            citecolor=blue,% The named colours include: black, white, red, yellow
            linkcolor=red,%                             blue, green, cyan, magenta
            anchorcolor=yellow,%
            filecolor=cyan,% Can set all all colors to a single color using the following command
            menucolor=red,% allcolors=<colour>.  For example, allcolors=black would be idea for
            runcolor=filecolor,% printing hard copies.  Simply uncomment the command on the following
            urlcolor=magenta,% line if you want this.
            %allcolors=black,
            pagebackref=true% In the bibliography gives citation page references. To turn on set
                            % to true
            ]{hyperref}

%% 
 % \usepackage[tagged, highstructure]{accessibility}
 % \usepackage{axessibility}
 %%
            
%
% Theorems and Remarks
%
\theoremstyle{definition}
\newtheorem{ass}{Assumption}

\theoremstyle{plain}
\newtheorem*{Thm}{Theorem}
\newtheorem{thm}{Theorem}
\theoremstyle{remark}
\newtheorem*{remark}{Remark}
\newtheorem{rem}{Remark}
\newtheorem{corr}{Corollary}
\theoremstyle{definition}
%\newtheorem{Def}{Definition}
\newtheorem*{deff}{Definition}

\newtheoremstyle{example}% name
  {9pt}%      Space above, empty = `usual value'
  {9pt}%      Space below
  {\normalfont} %\renewcommand{\myfiletype}{loe}}% Body font
  {}%         Indent amount (empty = no indent, \parindent = para indent)
  {\bfseries}% Thm head font
  {.}%        Punctuation after thm head
  {\newline}% Space after thm head: \newline = linebreak
  {\thmname{#1}\thmnumber{ #2.}\thmnote{ #3}}%Thm head spec
  
\theoremstyle{example}
\newtheorem{example}{Example}

\newtheoremstyle{Def}% name
  {9pt}%      Space above, empty = `usual value'
  {9pt}%      Space below
  {\normalfont}%\addcontentsline{lod}{Def}{\protect\numberline{\theDef}y}} %\renewcommand{\myfiletype}{lod}}% Body font
  {}%         Indent amount (empty = no indent, \parindent = para indent)
  {\bfseries}% Thm head font
  {.}%        Punctuation after thm head
  {\newline}% Space after thm head: \newline = linebreak
  {\thmname{#1}\thmnumber{ #2.}\thmnote{ #3}}%         Thm head spec
  
\theoremstyle{Def}
\newtheorem{Def}{Definition}

\newcommand{\abs}[1]{\ensuremath{\operatorname{abs}\left(#1\right)}}
\newcommand{\ans}{\newline\textit{\textcolor{red}{Answer:}}\newline}
\newcommand{\ant}{\textit{\textcolor{red}{Answer:}}\newline}
\makeatletter
   \def\argmax{\mathop{\operator@font argmax}}
   \def\argmin{\mathop{\operator@font argmin}}
\makeatother
%\newcommand{\argmax}{\arg\max}
%\newcommand{\argmin}{\arg\min}
\newcommand{\Beta}[2]{\ensuremath{\operatorname{B}\left(#1,#2\right)}}
\def\BibTeX{{\rm B\kern-.05em{\sc i\kern-.025em b}\kern-.08em
    T\kern-.1667em\lower.7ex\hbox{E}\kern-.125emX}}
\newcommand{\cmd}[1]{\mbox{\texttt{\symbol{'134}#1}}}
\newcommand{\Co}[1]{\ensuremath \mathrm{Cov}_{0}\left[#1\right]}
\newcommand{\cor}[2]{\ensuremath{\operatorname{cor}\left(#1,#2\right)}}
\newcommand{\cov}[2]{\ensuremath{\operatorname{Cov}\left[#1,#2\right]}}
\DeclareMathOperator{\diag}{diag}
\newcommand{\dr}{\ensuremath{\mrmd r}}
\newcommand{\ds}{\ensuremath{\mrmd s}}
\newcommand{\dt}{\ensuremath{\mrmd t}}
\newcommand{\dw}{\ensuremath{\mrmd w}}
\newcommand{\dx}{\ensuremath{\mrmd x}}
\newcommand{\dy}{\ensuremath{\mrmd y}}
\newcommand{\dz}{\ensuremath{\mrmd z}}
\DeclareMathOperator{\etr}{etr}
\newcommand{\E}[1]{\ensuremath{\operatorname{E}\left[#1\right]}}
\newcommand{\Eo}[1]{\ensuremath \mathrm{E}_{0}\left[#1\right]}
\newcommand{\Eq}[1]{\begin{align*}#1\end{align*}}
\newcommand{\Ex}[2]{\ensuremath{\operatorname{E}_{#1}\left[#2\right]}}
\newcommand{\Gammafrac}[2][2]{\ensuremath{\Gamma\left(\tfrac{#2}{#1}\right)}}%\Gamma[x]{y} yields Gamma(y/x)
\newcommand{\gfn}[1]{\ensuremath{\Gamma\left(#1\right)}}
\newcommand{\half}[1]{\ensuremath{\frac{#1}{2}}}
\newcommand{\hyperg}[2]{\ensuremath{\prescript{}{#1}{F}^{}_{#2}}}
\newcommand{\im}[1]{\ensuremath{\operatorname{Im}\left(#1\right)}}
\newcommand{\intinfty}{\int^{\infty}_{0}}
\newcommand{\matlab}{\textsc{Matlab}\xspace}
\newcommand{\mg}{\textit{\textcolor{red}{Marking Guide}}:\xspace\newline}
\newcommand{\mgt}{\newline\textit{\textcolor{blue}{Marking Guide:}}\newline}
\newcommand{\Mo}[1]{\ensuremath \mathrm{MSE}_{0}\left[#1\right]}
\newcommand{\mrmd}{\ensuremath{\,\mathrm{d}}}
\newcommand{\myrego}{\,\raisebox{1.5pt}{\scalebox{0.81}{\textregistered}}\xspace}
\DeclareMathOperator{\N}{N}
\newcommand{\ninf}[1][n]{#1\to\infty}
\newcommand{\param}[2][n]{\ensuremath{#2_{1} #2_{2}\ldots #2_{#1}}}
\newcommand{\params}[2][n]{\ensuremath{#2_{1},#2_{2},\ldots,#2_{#1}}}
\newcommand{\paramsj}[3][j]{\ensuremath{\left(#2_{1}\right)_{#1}\left(#2_{2}\right)_{#1}\ldots\left(#2_{#3}\right)_{#1}}}
\newcommand{\pdflatex}{pdf\LaTeX\xspace}
\makeatletter
   \def\plim{\mathop{\operator@font plim}}
\makeatother
\newcommand{\Poch}[2]{\ensuremath{\left(#1\right)_{#2}}}
\newcommand{\Prob}[1]{\ensuremath{\operatorname{Pr}\left(#1\right)}}
\newcommand{\re}[1]{\ensuremath{\operatorname{Re}\left(#1\right)}}
\newcommand{\sgn}[1]{\ensuremath{\operatorname{sgn}\left(#1\right)}}
%Duplicated commands:  Settle on a choice.  No. They do different things!
\newcommand{\solution}{\textit{\textcolor{red}{Solution}}:\xspace\newline}
\newcommand{\sol}{\textcolor{red}{Solution:}\xspace}
%
\DeclareDocumentCommand{\sumj}{ O{j} O{0} }{\ensuremath{\sum_{#1=#2}^{\infty}\,}}
\newcommand{\tod}{\ensuremath{\overset{d}{\to}}}
\DeclareMathOperator{\tr}{tr}
\DeclareMathOperator{\myvec}{vec}
\DeclareMathOperator{\vech}{vech}
% More duplicates: Choice as been made?
\newcommand{\V}[1]{\ensuremath{\operatorname{Var}\left[#1\right]}}
\newcommand{\Vhat}[1]{\ensuremath{\operatorname{\protect\widehat{V}}\left[#1\right]}}
\newcommand{\Vo}[1]{\ensuremath \mathrm{V}_{0}\left[#1\right]}
%\newcommand{\var}[1]{\ensuremath{\operatorname{Var}\left(#1\right)}}
\newcommand{\Vx}[2]{\ensuremath{\operatorname{Var}_{#1}\left[#2\right]}}

%Boxes
\newsavebox{\mystrutbox}
\savebox{\mystrutbox}{\rule{0mm}{0.6cm}}
\newcommand{\mystrut}{\usebox{\mystrutbox}}

\newlength{\twosuper}
\settowidth{\twosuper}{\textsuperscript{2}}

%
% List Management
%
\makeatletter
   \renewcommand{\p@enumii}{\relax}
   \def\plim{\mathop{\operator@font plim}}
\makeatother
\renewcommand{\theenumii}{(\alph{enumii})}
\renewcommand{\labelenumii}{\theenumii}
\makeatletter
   \renewcommand{\p@enumii}{\theenumi}
   \renewcommand{\p@enumiii}{\theenumi\theenumii}
\makeatother

% 
% An extensions to amsmath matrix environments
% (adds the same column specification options that array has)
% originally from http://texblog.net/latex-archive/maths/amsmath-matrix/
% 
\makeatletter
\renewcommand*\env@matrix[1][*\c@MaxMatrixCols c]{%
  \hskip -\arraycolsep
  \let\@ifnextchar\new@ifnextchar
  \array{#1}}
\makeatother


\thispagestyle{empty}
\bibliographystyle{Skeels}

\title{ECOM40006/ECOM90013 Econometrics 3 \\
        Department of Economics \\
        University of Melbourne}
\date{Semester 1, \the\year}



\title{Week 1  Tutorial Exercise Solutions}

\begin{document}

\maketitle
\enlargethispage{\baselineskip}
\thispagestyle{empty}
\begin{enumerate}
\item 
\begin{enumerate}
\item Download LaTeX.zip from the LMS and expand the archive (if
that doesn't happen automatically).  Inside the folder that is
created is another folder called `Assignment Template', which
contains a file called assignment.tex.  Open this file in a text
editor.  (Specifically, do not use Word or any other editor that
will insert hidden material in the file.  A plain text editor
like Notebook (on Windows) or TextEdit (on a Mac), or better
still a LaTeX-aware editor of some description.)
      
\item Having opened assignment.tex you will see that there is a set of
places where you need to insert indformation about yourself, your
subject, and the task at hand, e.g.\ Exercise 1.  Enter these details
into your file and then save it with a name other than assignment.tex.
It is helpful to use names that allow you to discern what is inside
the file without opening it.  For example, ECOM3-Ex1.tex might work
for me.  You use whatever name you wish.
      
\item From the Library catalogue, obtain the Bib\TeX\ entry
corresponding to \citet{abadir99}.  Also, manually create the Bib\TeX\
entry for \citet{rao}.  Add both entries to a .bib file and include
the contents of the .bib file in your report.  (You may want to
explore the use of the \texttt{listings} package to do this.)
      
\item Do a web search for knitr.  (Yes, that is how it is
spelt.)  If you end up at the Wikipedia page then follow the
link to
\url{https://github.com/yihui/knitr-examples/blob/master/002-minimal.Rnw}
to see a very basic example of it's use.  Convince yourself that
you can run your favourite regression in R and incorporate the
results using knitr.
\end{enumerate}

\item Use \LaTeX, strictly pdf\LaTeX, to re-create the pdf document
containing this Tutorial Exercise.  You must create a .bib file for
the references and include the text of that file in your report.  (You
may want to explore the use of the \texttt{listings} package to do
this.).  If using Bib\TeX\ then you probably need to read the
documentation for the \texttt{natbib} package to see how to cite
material.  You may also need to use the \texttt{url} package to add
the URL or the \texttt{hyperref} package.

\end{enumerate}

\bibliography{theNameOfYourBibFile} 
% My Bib file is called CHRIS.bib, so I would just put CHRIS here. 

\end{document}

\end{lstlisting}

\end{document}
